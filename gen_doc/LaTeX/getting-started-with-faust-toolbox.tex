\documentclass[a4paper,11pt]{report}
\author{Adrien Leman; Nicolas Bellot}
\title{Getting Started with FA$\mu$ST Toolbox \smallbreak "Flexible Approximate Multi-Layer Sparse Transforms" \bigbreak
\bigbreak
\centerline{\includegraphics[scale=0.1]{images/logo.png}}
}

\setcounter{secnumdepth}{3}
\setcounter{tocdepth}{2}


\usepackage[]{mcode} % import matlab file
\usepackage[utf8]{inputenc}
\usepackage[T1]{fontenc}
\usepackage[margin=1in]{geometry}
\usepackage{graphicx, subcaption, enumerate}
\usepackage{amsmath}
\usepackage{amsfonts}
%\usepackage{amsthm}
\usepackage[boxruled,algo2e]{algorithm2e}

\SetCommentSty{mycommfont}

\usepackage{algorithmic}
\usepackage{comment}
\usepackage[dvipsnames]{xcolor}
\usepackage{enumitem}
\usepackage{url}
\usepackage{multirow}
\usepackage{sectsty}
\usepackage[font=small,labelfont=bf]{caption}
\usepackage{placeins}
\usepackage{epstopdf}
\usepackage{hyperref}

\hypersetup{
	colorlinks=true, 
	linkcolor=red, % color of internal links (change box color with linkbordercolor)
    citecolor=green,        % color of links to bibliography
    filecolor=magenta,      % color of file links
    urlcolor=blue 
}

\usepackage{tikz}
%\usepackage{appendix}
\usepackage{mathtools}

% pour afficher les figure Here
\usepackage{float}

\usepackage{listings}
\usepackage{color} %red, green, blue, yellow, cyan, magenta, black, white
\definecolor{mygreen}{RGB}{28,172,0} % color values Red, Green, Blue
\definecolor{mylilas}{RGB}{170,55,241}

\definecolor{mygray}{rgb}{0.95,0.95,1}
\definecolor{mygrayMatlab}{rgb}{0.9,0.9,0.9}
\definecolor{mymauve}{rgb}{0.58,0,0.82}
\definecolor{myBlack}{rgb}{0.3,0.3,0.3}

% Set your language (you can change the language for each code-block optionally)

\lstdefinestyle{customBash}{
  language=bash,%  
  backgroundcolor=\color{myBlack},   % choose the background color; you must add 
  stringstyle=\color{mylilas},%
  %identifierstyle=\bfseries\color{white},%
  keywordstyle=\color{white},%
  basicstyle=\ttfamily\color{white},
  commentstyle=\color{mygreen},%
  %showstringspaces=true,%without this there will be a symbol in the places where there is a space
  %basicstyle=\ttfamily,        		% the size of the fonts that are used for the code
%  breaklines=true,                 	% sets automatic line breaking
%  deletekeywords={...},            % if you want to delete keywords from the given language
  frame=single,	                   % adds a frame around the code
  keepspaces=true,                 % keeps spaces in text, useful for keeping indentation of code (possibly needs columns=flexible)
  numbers=left,                    % where to put the line-numbers; possible values are (none, left, right)
  numbersep=5pt,                   % how far the line-numbers are from the code
  numberstyle=\tiny\color{gray}, % the style that is used for the line-numbers
  rulecolor=\color{black},         % if not set, the frame-color may be changed on line-breaks within not-
}


\lstdefinestyle{customC}{
  backgroundcolor=\color{mygray},   % choose the background color;
  belowcaptionskip=1\baselineskip,
  breaklines=true,
  frame=single,	                   % adds a frame around the code
  %frame=L,
  xleftmargin=\parindent,
  language=C,
  showstringspaces=false,
  basicstyle=\footnotesize\ttfamily,
  keywordstyle=\bfseries\color{green!40!black},
  commentstyle=\itshape\color{purple!40!black},
  identifierstyle=\color{blue},
  stringstyle=\color{orange},
}

\lstdefinestyle{customMatlab}{
	language=Matlab,% 
	backgroundcolor=\color{mygray},%
	stringstyle=\color{mylilas},%
  	%identifierstyle=\bfseries\color{white},%
  	basicstyle=\ttfamily\color{black},
  	commentstyle=\color{mygreen},%
    morekeywords={matlab2tikz},%
    keywordstyle=\color{black},%
    morekeywords=[2]{1}, keywordstyle=[2]{\color{black}},%
    %identifierstyle=\color{black},%
    stringstyle=\color{mylilas},%
    commentstyle=\color{mygreen},%
	frame=single,	                   % adds a frame around the code
	breaklines=true,%
    showstringspaces=false,%without this there will be a symbol in the places where there is a space
    numbers=left,%
    numberstyle={\tiny \color{black}},% size of the numbers
    numbersep=9pt, % this defines how far the numbers are from the text
    %emph=[1]{for,end,break},emphstyle=[1]\color{red}, %some words to emphasise
    %emph=[2]{word1,word2}, emphstyle=[2]{style},  
}


% bold symbols
\newcommand{\bfA}{\mathbf{A}}
\newcommand{\bfX}{\mathbf{X}}
\newcommand{\bfW}{\mathbf{W}}
\newcommand{\bfD}{\mathbf{D}}
\newcommand{\bfQ}{\mathbf{Q}}
\newcommand{\bfR}{\mathbf{R}}
\newcommand{\bfM}{\mathbf{M}}
\newcommand{\bfw}{\mathbf{w}}
\newcommand{\bfz}{\mathbf{z}}
\newcommand{\bfx}{\mathbf{x}}
\newcommand{\bfr}{\mathbf{r}}
\newcommand{\bfe}{\mathbf{e}}
\newcommand{\bfy}{\mathbf{y}}
\newcommand{\hbfz}{\hat{\mathbf{z}}}
\newcommand{\bfI}{\mathbf{I}}
\newcommand{\bfU}{\mathbf{U}}
\newcommand{\bfu}{\mathbf{u}}
\newcommand{\bfV}{\mathbf{V}}
\newcommand{\bfv}{\mathbf{v}}


% sums
\newcommand{\sump}{\sum_{p=1}^P}
\newcommand{\summ}{\sum_{j=1}^m}
\newcommand{\sumk}{\sum_{k=1}^K}
\newcommand{\sumt}{\sum_{t=1}^T}
\newcommand{\sumko}{\sum_{k_1=1}^K}
\newcommand{\sumkt}{\sum_{k_2=1}^K}
\newcommand{\sumn}{\sum_{i=1}^N}


% general
\newcommand{\dataset}{\mathcal{X}}
\newcommand{\fhalf}{\frac{1}{2}}
\newcommand{\neucl}[1]{\left\lVert#1\right\rVert_2}
\newcommand{\ipeucl}[2]{\left\langle #1,#2 \right\rangle_2}
\newcommand{\smeas}{\mathcal{M}}
\newcommand{\pmeas}{\mathcal{P}}
\newcommand{\gaussset}{\mathcal{G}}
\newcommand{\gaussmix}[1]{{\mathcal{G}_{#1}}}
\newcommand{\sph}[1]{\mathcal{S}_{#1-1}}
\newcommand{\fracsqm}{\frac{1}{\sqrt{m}}}
\newcommand{\sqm}{\sqrt{m}}
\newcommand{\argmin}[1]{\underset{#1}{\text{argmin}}}


% sketches
\newcommand{\freqs}{\Omega} %frequencies
\newcommand{\freq}{{\boldsymbol\omega}} %frequency
\newcommand{\scalfreq}{\omega}
\newcommand{\hyppar}{\Theta} %hyper parameter
\newcommand{\chrc}{\psi} %characteristic function
\newcommand{\skop}{\mathcal{A}} %sketching operator
\newcommand{\malpha}{{\boldsymbol\alpha}} %weights (to rename)
\newcommand{\mmu}{{\boldsymbol\mu}} %means
\newcommand{\mSigma}{{\boldsymbol\Sigma}} %variances
\newcommand{\msigma}{{\boldsymbol\sigma}} %diag variances
\newcommand{\mrho}{{\boldsymbol\varphi}} %frequency angle
\newcommand{\mtheta}{{\boldsymbol\theta}} %frequency radius
\newcommand{\relsprs}{\rho} %relative sparsity (for results)
\newcommand{\relmsrm}{\delta} %relative number of parameters (for results)
\newcommand{\thetaset}{\mathcal{T}}
%\newcommand{\thetaset}{\Theta}
\newcommand{\Nsmall}{{N_{small}}}
\newcommand{\msmall}{{m_{small}}}
\newcommand{\gmmcover}{\Gamma}
\newcommand{\pp}{P}
\newcommand{\PP}{\mathbb{P}}
\newcommand{\qq}{Q}
\newcommand{\dens}{p}
\newcommand{\cstdom}{\mathrm{A}}

% rkhs
\newcommand{\rkhs}{\mathcal{H}}
\newcommand{\ipH}[2]{\left\langle #1,#2 \right\rangle_\rkhs}
\newcommand{\normH}[2]{\gamma_k\left(#1,#2\right)}
\newcommand{\normHnota}{\gamma_k}
\newcommand{\featm}{\phi}
\newcommand{\freqdist}{\Lambda}
\newcommand{\freqdistn}{\hat{\Lambda}}
\newcommand{\model}{\Sigma}
\newcommand{\Xspace}{X}
\newcommand{\TIk}{\mathbf{K}}
\newcommand{\supp}{\text{\upshape supp}}
\newcommand{\secant}{\mathcal{S}}
\newcommand{\decod}{\Delta}
\newcommand{\enet}{\mathcal{N}}
\newcommand{\kernel}{\kappa}
\newcommand{\meas}{\nu} % to avoid mu, which is the mean of gaussian
\newcommand{\mmap}{\varphi}
\newcommand{\sigmaker}{\sigma_\freqdist}
\newcommand{\sigkersmall}{S}

%imaginary unit
\newcommand{\imaginaryi}{\mathsf{i}}

% misc
\definecolor{light-gray}{gray}{0.3}
\newcommand{\mycomment}[1]{\emph{\small \color{red} NK: #1}}
\newcommand{\mytodo}[1]{\emph{\color{blue} #1}}

% io proba
\newcommand{\bigprbsp}{\mathbf{S}}
\newcommand{\event}{E}
\newcommand{\eventset}{\mathcal{F}}
\newcommand{\prbfc}{\mathbb{P}}
\newcommand{\measop}{\mathbf{M}}
\newcommand{\eps}{\epsilon}
\newcommand{\outc}{s}
\newcommand{\decop}{\mathbf{y}'}
\newcommand{\skopprb}{\skop^{(\outc)}}
\newcommand{\pprb}{\hat{\pp}^{(\outc)}}
\newcommand{\pproj}{\pp_\text{proj}}

\newcommand{\distfun}[3]{\gamma_{#3}\left(#1,#2\right)}
\newcommand{\distfunsq}[3]{\gamma^2_{#3}\left(#1,#2\right)}
\newcommand{\distfuns}[1]{\gamma_{#1}}
\newcommand{\normE}[2]{\distfun{#1}{#2}{E}}
\newcommand{\normEs}{\distfuns{E}}
\newcommand{\normF}[2]{\distfun{#1}{#2}{F}}
\newcommand{\normFs}{\distfuns{F}}
\newcommand{\normX}[2]{\distfun{#1}{#2}{X}}
\newcommand{\normXs}{\distfuns{X}}
\newcommand{\normM}[2]{\distfun{#1}{#2}{M}}
\newcommand{\normMs}{\distfuns{M}}
\newcommand{\normG}[2]{\distfun{#1}{#2}{G}}
\newcommand{\normGs}{\distfuns{G}}
\newcommand{\normmix}[2]{\distfun{#1}{#2}{mix}}
\newcommand{\normmixs}{\distfuns{mix}}
\newcommand{\normK}[2]{\distfun{#1}{#2}{\freqdist}}
\newcommand{\normKs}{\distfuns{\freqdist}}
\newcommand{\normKsq}[2]{\distfunsq{#1}{#2}{\freqdist}}
\newcommand{\normker}[2]{\distfun{#1}{#2}{\kernel}}
\newcommand{\normkers}{\distfuns{\kernel}}
\newcommand{\normkersq}[2]{\distfunsq{#1}{#2}{\kernel}}

\newcommand{\diam}[1]{\text{\upshape diam}\left(#1\right)}
\newcommand{\rad}[1]{\text{\upshape radius}\left(#1\right)}

%\newcommand{\LRIPsu}{\text{LRIP}}
%\newcommand{\BPnu}{\text{BP}}
%\newcommand{\CIOPsu}{\text{CIOP}}
%\newcommand{\prop}{\Pi}
%\newcommand{\nSig}[2]{\|#1\|_{\sigma_{#2}^2}}
%\newcommand{\ipW}[2]{\left\langle #1,#2 \right\rangle_2}

\newcommand{\embd}{\phi}

\makeatletter
\newcommand{\nosemic}{\renewcommand{\@endalgocfline}{\relax}}% Drop semi-colon ;
\newcommand{\dosemic}{\renewcommand{\@endalgocfline}{\algocf@endline}}% Reinstate semi-colon ;
\newcommand{\pushline}{\Indp}% Indent
\newcommand{\popline}{\Indm\dosemic}% Undent
\makeatother

\newcommand{\code}[1]{\texttt{#1}}



\begin{document}

\maketitle

\tableofcontents
\newpage


%!TEX root =  gettingStartedFAuST-version2_0.tex
\chapter{Introduction}\label{sec:intro}

\paragraph{What is the \FAuST\ toolbox ?} FA$\mu$ST is a C++ toolbox, useful to decompose or approximate any given dense matrix into a product of sparse matrices, in order to reduce its computational complexity (both for storage and manipulation). 
As an example, Figure \ref{fig:presentation} shows a dens matrix \textbf{A} and sparse matrices $\mathbf{S_j}$ such that $\mathbf{A}=\prod_{j=1}^J\mathbf{S_j}$.

\begin{figure}[H] %%[!htbp]
\centering
\includegraphics[scale=0.5]{images/hadamard32_bw.pdf}
\caption{Brief presentation of FA$\mu$ST}
\label{fig:presentation}
\end{figure}

\paragraph{Why shall I use \FAuST\ ?} FA$\mu$ST can be used to speed up iterative algorithms commonly used for solving high dimensional linear inverse problems. The algorithms implemented in the toolbox are described in details by Le Magoarou \cite{LeMagoarou2016}. For more information on the FA$\mu$ST project, please visit the website of the project: \url{http://faust.gforge.inria.fr}.


%\paragraph{Brief description:} 
%$A=\prod_{j=1}^J S_j$.

%Valid Installation: Platforms, Compiler and IDE
\paragraph{Is my platform and compiler currently supported ?} 
\FAuST\ has been tested in various configurations, and is currently delivered with a Matlab wrapper. Figure \ref{fig:recapInstall} summarizes the configurations on which \FAuST\ has been tested. 

\paragraph{How should I use this document ?}
If your platform and configuration is currently supported according to Figure \ref{fig:recapInstall}, please refer to the corresponding Install Chapter. By default we suggest the installation using the GCC or Clang compiler directly from a command terminal since it requires fewer externals components. 
\begin{itemize}
\item for installation on UNIX (including Linux and Mac OS X) platform refer to Chapter \ref{sec:InstallUnix};
\item for installation on Windows refer to Chapter \ref{sec:WinInstall}. 
\end{itemize}
Chapter \ref{sec:firstUse} shows quickly how to use this library and gives an example of application. Finally, a "work in progress" part is given Chapter \ref{sec:WorkingProgress} to give an overview of the roadmap for \FAuST. 

\begin{figure}[H] %%[!htbp]
\centering
\includegraphics[scale=0.4]{images/recapInstall_v2-1.pdf}
\caption{Tested configurations: Platforms and IDE}
\label{fig:recapInstall}
\end{figure}

\paragraph{How is the code of \FAuST structured ?}
A brief outline of the code structure and status is given on Figure~\ref{fig:faustStructure}. The main C++ library called \textrm{libfaust} includes two components:
\begin{itemize}
\item the "FA$\mu$ST matrix multiplication" provides efficient linear operators to efficiently manipulate your data, relying on efficient low-level routines from external libraries where needed; 
\item the "Factorization algorithms" is used to generate a FA$\mu$ST core from a dense matrix. 
\end{itemize}
Various wrappers are planned to call \textrm{libfaust}. In the current version of \FAuST\ (Version 2.1), only the Matlab wrapper and the Python wrapper is implemented. Command-line and A||GO wrapper are planned for the next version of FA$\mu$ST (see Section \ref{sec:WorkingProgress}).   


\begin{figure}[H] %%[!htbp]
\centering
\includegraphics[scale=0.45,trim = 0cm 3cm 0cm 3cm, clip]{images/FaustStructure2-1.pdf}
\caption{Brief structure of FA$\mu$ST version 2.1.0}
\label{fig:faustStructure}
\end{figure}

\paragraph{Known issues :}
\begin{enumerate}
\item The installation on Windows using Visual Studio IDE has been tested with success in certain configurations, however \textbf{compilation problems have been encountered depending on the version of Windows and Visual Studio}. Unfortunately we cannot offer technical support but you may nevertheless report problems and/or suggestions on the mailing list \url{http://lists.gforge.inria.fr/pipermail/faust-install/}. 
 
\item Compiling mex files with the \textbf{mex compiler} of Matlab requires a compatible third-party compiler. \textbf{Warning: With  Matlab Release2016a, the mex compiler seems to only support up to GCC 4.7 (see \url{http://fr.mathworks.com/support/compilers/R2016a/index.html?sec=glnxa64} for more detail)}.

\item The "Factorization algorithms" module represented on Figure \ref{fig:faustStructure} is still work in progress (see Section \ref{sec:WorkingProgress}). A GPU implementation is also on its way, as well as more wrappers.

\end{enumerate}

\paragraph{How do I pronounce \FAuST\ ?} We suggest to pronounce the library name as ``FAUST'', but you may also pronounce it ``FAMUST'' ;-)


\paragraph{License :}Copyright (2016) Luc Le Magoarou, Remi Gribonval INRIA Rennes, FRANCE \\
The FA$\mu$ST Toolbox is distributed under the terms of the GNU Affero General Public License. This program is free software: you can redistribute it and/or modify it under the terms of the GNU Affero General Public License as published by the Free Software Foundation. This program is distributed in the hope that it will be useful, but WITHOUT ANY WARRANTY; without even the implied warranty of MERCHANTABILITY or FITNESS FOR A PARTICULAR PURPOSE.  See the GNU Affero General Public License for more details. You should have received a copy of the GNU Affero General Public License along with this program.  If not, see \url{http://www.gnu.org/licenses/}.


\section{New Features since last version}\label{sec: RequiredTools}

Here are presented somes new features that are incorporated in the current version (2.1) and weren't available in the former version.

\paragraph{Python wrapper}
The a beta version of the Python wrapper is now available.
The list of components you musthave to use this wrapper is defined in the following section (\ref{sec:Pythonwrapper}).



\paragraph{Matlab wrapper : complex scalar compatibility}
The Matlab wrapper is now compatible with complex scalar.
This allows you to construct Faust from complex factors
and multiply with complex matrices.
Nevertheless, the transconjugate and conjugate operation are not yet implemented. 

\paragraph{C++ code : improvement}
Some improvement have been made to improve the speed-up of a Faust. For instance, a factor of a Faust is no longer necessarily a sparse matrix.Now, it's a  now generic matrix.
This means that currently it can be a full-storage matrix or sparse one. The choice of the format that will represent the factor is made at execution time for more flexibility.
Another advantage, this for future development, the class/interface of generic matrices can be inherited/implemented by more specific format (block-diagonal matrices, permutation matrices) or more efficient. 
\chapter{Installation on Linux and MAC OS X platform}\label{sec:InstallUnix}

\paragraph{}The FA$\mu$ST project is based on C++ library available for both UNIX and Windows environments. The proposed toolbox provides a Matlab wrapper. CMake has been chosen to build the project FA$\mu$ST because it is an open-source, cross-platform family of tools designed to build, test and package software. This chapter presents the steps to install the FA$\mu$ST tools on Unix platform (both Linux and Mac OS). 

\paragraph{}Please ensure that the \textbf{prerequisites components} listed in Section \ref{sec:RequiredTools} are installed. Then refer to the appropriate section : 
\begin{itemize}
\item Basic installation using \textbf{the command line terminal}, refer to Section \ref{sec:UnixBuildInstall}.
\item Basic installation using \textbf{the IDE "Code::Blocks"} for both Linux and MAC OS X, refer to Section \ref{sec:UnixInstallCodeBlock}. 
\item Basic installation using \textbf{the IDE "Xcode"} only on MAC OS X, refer to Section \ref{sec:MacInstallXcode}. 
\end{itemize}

Finally in Section \ref{sec:UnixCustomInstall}, the configure options available to build the FA$\mu$ST toolbox are described to propose an Custom - Advanced installation. For example, optional configuration can be activate such as modifying the install directory, or building in debug mode).  

\section{Required components}\label{sec:RequiredTools}
This Section lists the components required for FA$\mu$ST installation. 
\begin{itemize}
\item \textbf{Install CMake}. Visit the website \url{https://cmake.org/} to process to the installation.
\item \textbf{Verify Cmake install} by typing in a command terminal : 
\begin{lstlisting}
which cmake
\end{lstlisting}
The command terminal returns the path of your Cmake binary file like :
\begin{lstlisting}
/usr/bin/cmake
\end{lstlisting}
If not, add Cmake binary directory in the environment path. (in your ~/.bashrc file)

\item \textbf{Install Matlab} (\url{https://fr.mathworks.com/downloads/})

\item \textbf{Verify Matlab install} by typing in a terminal the following command : 
\begin{lstlisting}
which matlab
which mex
\end{lstlisting}
You must obtain the path of your matlab and mex binaries files like: 
\begin{lstlisting}
/usr/local/bin/matlab
/usr/local/bin/mex
\end{lstlisting}
If not, add \texttt{matlab} and \texttt{mex} binaries directories in your environment path (in your ~/.bashrc file). 

\item \textbf{export CC and CXX variables} corresponding to gcc and g++ binaries path :\\
find your gcc and g++ path using \texttt{which} command in a terminal :
\begin{lstlisting}
which gcc
which g++
\end{lstlisting}

Open your ~/.bashrc file and save the return-path of gcc and g++ like : \\
\begin{lstlisting}
# export version of gcc
export CC=/usr/lib64/ccache/gcc
export CXX=/usr/lib64/ccache/g++
\end{lstlisting}
\end{itemize}


\section{Basic build and installation}\label{sec:UnixBuildInstall}
\paragraph{}When prerequisities listed in precedent section \ref{sec:RequiredTools} are checked, the FA$\mu$ST installation can be done : 
if you are administrator of your machine (root access), follow instructions given in section \ref{sec:UnixBuildInstallAdmin}. Otherwise, for an local installation, refers to the section \ref{sec:UnixBuildInstallNOAdmin}. 

\subsection{Install with administrator privilege}\label{sec:UnixBuildInstallAdmin}

\begin{itemize}
\item \textbf{Download} the FA$\mu$ST package on the website :  \url{http://faust.gforge.inria.fr/}
\item \textbf{Open} a command terminal
\item \textbf{Set the current directory} to your FA$\mu$ST directory (NOTE: do not use any special character in your FAUST directory path, for example the character $\mu$)
\item Type the following commands : 
\begin{lstlisting}
mkdir build
cd build
cmake ..
make
sudo make install % run with administrator privilege
\end{lstlisting}
\end{itemize}

\paragraph{INFO:}When using the \textbf{cmake} command to generate the build system, \textbf{cmake} performs a list of tests to determine the system configuration and manage the build system. If the configuration is correct then the build system is generated and written. In this case, the three last lines of the console log of \textbf{cmake} command should be:
\begin{lstlisting}
-- Configuring done 
-- Generating done 
-- Build files have been written to: <YOUR/LOCAL/DIRECTORY/build>
\end{lstlisting}

\paragraph{INFO:}The command \textbf{make} will compile the build files.

\paragraph{INFO:}The command \textbf{sudo make install} will install the library and others components in the default directory: \\
\texttt{/usr/local/lib/libfaust.a} for the faust library, \\
\texttt{\textasciitilde /Documents/MATLAB/faust/} for the wrapper matlab.
\\
You must have administrator privilege because the library file \texttt{libfaust.a} is copied in an root path directory. If you do not have administrator privilege, you can make a local install (see. section \ref{sec:UnixBuildInstallNOAdmin}).

\subsection{Install without administrator privilege}\label{sec:UnixBuildInstallNOAdmin}

\begin{itemize}
\item \textbf{Download} the FA$\mu$ST package on the website :  \url{http://faust.gforge.inria.fr/}
\item \textbf{Open} a command terminal
\item \textbf{Set the current directory} to your FA$\mu$ST directory (NOTE: do not use any special character in your FAUST directory path, for example the character $\mu$)
\item Type the following commands : 
\begin{lstlisting}
mkdir build
cd build
cmake .. -DCMAKE_INSTALL_PREFIX="<Your/Install/Dir>"
make
make install
\end{lstlisting}
\end{itemize}

\paragraph{INFO:}When using the \textbf{cmake} command to generate the build system, \textbf{cmake} performs a list of tests to determine the system configuration and manage the build system. If the configuration is correct then the build system is generated and written. In this case, the three last lines of the console log of \textbf{•}{cmake} command should be: \begin{lstlisting}
-- Configuring done 
-- Generating done
-- Build files have been written to: <YOUR/LOCAL/DIRECTORY/build>
\end{lstlisting}

\paragraph{INFO:}The command \textbf{make} will compile the build files.

\paragraph{INFO:}The command \textbf{make install} will install the library and others components in the defined install directory. 






% CODEBLOCKS
\section{Build \& Install using Code Block}\label{sec:UnixInstallCodeBlock}
\paragraph{}When prerequisities listed in precedent section \ref{sec:RequiredTools} are checked, the FA$\mu$ST installation can be done : 

\begin{itemize}
\item \textbf{Download} the FA$\mu$ST package on the website :  \url{http://faust.gforge.inria.fr/}
\item \textbf{Open} a command terminal
\item \textbf{Set the current directory} to your FA$\mu$ST directory (NOTE: do not use any special character in your FAUST directory path, for example the character $\mu$)
\item Type the following commands : 
\begin{lstlisting}
mkdir build
cd build
cmake .. -G "CodeBlocks - Unix Makefiles" -DCMAKE_INSTALL_PREFIX="<Your/Install/Dir>"
\end{lstlisting}

\item Open the FAUST project from the file \textbf{./build/FAUST.cbp} with Code::Blocks IDE. 
\item In Code::Blocks IDE, select \textbf{ALL} target and build the project. 
\item In Code::Blocks IDE, select \textbf{install} target and build the project. 
\end{itemize}

\paragraph{INFO:}When using the \textbf{cmake} command to generate the build system, \texttt{cmake} performs a list of tests to determine the system configuration and manage the build system. If the configuration is correct then the build system is generated and written. In this case, the three last lines of the console log of \texttt{cmake} command should be:
\begin{lstlisting}
-- Configuring done 
-- Generating done 
-- Build files have been written to: <YOUR/LOCAL/DIRECTORY/build>
\end{lstlisting}
The optional parameter \texttt{$-G "CodeBlocks - Unix Makefiles"$} allows to generate the Code Blocks project and the Unix Makefiles. The optional parameter -DCMAKE\_INSTALL\_PREFIX="<Your/Install/Dir>" allows to install the binaries on the selected directory. If not used, the \texttt{make install} process must be realized with administrator privilege because the library file \texttt{libfaust.a} is copied in an root path directory. 


\section{Build \& Install using Xcode (for MAC OS)}\label{sec:MacInstallXcode}

The building with IDE Xcode concerns only MAC OS X environment.
\paragraph{}When prerequisities listed in precedent section \ref{sec:RequiredTools} are checked, the FA$\mu$ST installation can be done : 
\begin{itemize}
\item \textbf{Download} the FA$\mu$ST package on the website :  \url{http://faust.gforge.inria.fr/}
\item \textbf{Open} a command terminal
\item \textbf{Set the current directory} to your FA$\mu$ST directory (NOTE: do not use any special character in your FAUST directory path, for example the character $\mu$)
\item Type the following commands : 
\begin{lstlisting}
mkdir build
cd build
cmake .. -G "Xcode" -DCMAKE_INSTALL_PREFIX="<Your/Install/Dir>"
\end{lstlisting}

\item Open the FAUST project from the file \textbf{./build/FAUST.xcodeproj} with Xcode IDE. 
\item In Xcode IDE, select \textbf{ALL} target and build the project. 
\item In Xcode IDE, select \textbf{install} target and build the project. 
\end{itemize}

\paragraph{INFO:}When using the \textbf{cmake} command to generate the build system, \texttt{cmake} performs a list of tests to determine the system configuration and manage the build system. If the configuration is correct then the build system is generated and written. In this case, the three last lines of the console log of \texttt{cmake} command should be:
\begin{lstlisting}
-- Configuring done 
-- Generating done 
-- Build files have been written to: <YOUR/LOCAL/DIRECTORY/build>
\end{lstlisting}
The optional parameter \texttt{$-G "Xcode"$} allows to generate the XCode project. The optional parameter -DCMAKE\_INSTALL\_PREFIX="<Your/Install/Dir>" allows to install the binaries on the selected directory. If not used, the \texttt{make install} process must be realized with administrator privilege because the library file \texttt{libfaust.a} is copied in an root path directory. 


\paragraph{NOTE:}You can generated the target using the terminal command \texttt{xcodebuild} :
\begin{lstlisting}
mkdir build
cd build
cmake .. -G "Xcode"
%% list all target of the project
xcodebuild -list -project FAUST.xcodeproj 
%% Build the targets
xcodebuild -configuration "Release" -target "ALL_BUILD" build 
%% performs the "make install"
xcodebuild -configuration "Release" -target "install" build 
\end{lstlisting}




\section{Custom - Advanced Installation}\label{sec:UnixCustomInstall}

\paragraph{}The project FA$\mu$ST can be configured with optional parameters, for example if you want to install FA$\mu$ST in a different folder or to enable the parallel computing using multithread capacities provided by the OS. This build system can be parametrized using the Cmake Graphical User Interface, or the Cmake command line tools. 

\paragraph{}The Cmake Graphical User Interface ccmake allows you to select option input. When using the \texttt{ccmake} command in your build directory, the Cmake GUI appears in the console (see fig. \ref{fig:ccmake}).

\begin{figure}[!h] %%[!htbp]
\centering
\includegraphics[scale=0.5]{images/ccmake.jpg}
\caption{ccmake GUI}
\label{fig:ccmake}
\end{figure}


%\section{Optional dependent tools}\label{sec:OptionalRequiredTools}
\paragraph{Optional Install of GPU process}
\begin{itemize}
\item \textbf{Install} CUDA and the drivers for NVIDIA.
\item \textbf{Verify install} of GPU tools by typing in a terminal : 
\begin{lstlisting}
which nvcc
\end{lstlisting}
You must obtain the path of your \texttt{nvcc} compiler like 
\begin{lstlisting}
/usr/local/cuda-7.5/bin/nvcc
\end{lstlisting}
If not, add \texttt{nvcc} directory in your environment path (in your ~/.bashrc file). 

\end{itemize}



\paragraph{}When scrolling on a value and pressing [enter], this value can be edited, the black underlaid row displays some information about the option and required path to create the build system. In the case of an option press [enter] to toggle the ON/OFF values. You can edit option by pressing [enter]. For example, press [enter] to edit option \texttt{CMAKE\_INSTALL\_PREFIX} to modify the install directory. 
\paragraph{}After choosing options for the build and setting the required fields, press [c] to configure. The configuration of the build system is checked again by Cmake, at the end of this check if the build settings are correct, you can press [g] in order to generate the build system.

\paragraph{} Instead the ccmake GUI, an other possibility to configure and generate the project is to use the command line cmake which can take the option input. Here is the list of available options: 
\texttt{$cmake\ ..\ -D<BUILD\_NAME>=<value>$}

\begin{itemize}
\item CMAKE\_INSTALL\_PREFIX : Install directory
\item CMAKE\_INSTALL\_MATLAB\_PREFIX : Install directory for the Matlab wrapper (mexfunction, demo ...)
\item BUILD\_TESTING : Enable the ctest option (default value is ON)
\item BUILD\_DOCUMENTATION : Generating the doxygen documentation (default value is OFF)  
\item BUILD\_MULTITHREAD : Enable multithread with OpenMP Multithreading (default value is OFF)
\item BUILD\_VERBOSE : Enable verbose option when compile (-v) (default value is OFF)
\item BUILD\_DEBUG : Enable FAUST Debug mode (default value is OFF )
\item BUILD\_USE\_GPU : Using both CPU and GPU process ( default value is OFF)
\item BUILD\_MATLAB\_MEX\_FILES : Enable building Matlab MEX files (default value is ON)
\item BUILD\_OPENBLAS : Using openBLAS for matrix and vector computations (default value is OFF )
\item BUILD\_READ\_XML\_FILE : Using xml2 library to read xml files (default value is OFF)
\item BUILD\_READ\_MAT\_FILE : Using matio library to read mat files (default value is OFF)
\end{itemize}

\paragraph{}Following the selected option, the cmake installer automatically checks the dependent component (library OpenBlas, eigen, matio, libxml2).  








\subsection{Required packages}\label{sec:RequiredPackages}

\paragraph{}Here is a list of packages used in the FA$\mu$ST project. The installation of this packages are automatically done. There are nothing to do. (see the directory "./externals").
\begin{itemize}
\item Library eigen \url{http://eigen.tuxfamily.org}
\item Library openBlas \url{http://www.openblas.net}
\item Library xml2 \url{http://xmlsoft.org}
\item Library matio \url{https://sourceforge.net/projects/matio}
\end{itemize}

\paragraph{Compatibility between MATLAB and compiler gcc} \textbf{Adjust your version of GCC compiler} in order to run the installation properly. The use of the mex function in Matlab requires that you have a third-party compiler installed on your system. The latest version of Matlab (2016a in our case) only supports up to GCC 4.7 (see \url{http://fr.mathworks.com/support/compilers/R2016a/index.html?sec=glnxa64} for more detail).

\chapter{Installation on Windows platform}\label{sec:WinInstall}


\paragraph{}The FA$\mu$ST project is based on \textbf{C++ library} available for both UNIX and Windows environments. The proposed toolbox provides a Matlab wrapper. \textbf{CMake} has been chosen to build the project FA$\mu$ST because it is an open-source, cross-platform family of tools designed to build, test and package software. This chapter presents the steps to install the FA$\mu$ST tools on Windows platform.

\paragraph{}Firstly, please ensure that the \textbf{prerequisites components} listed in Section \ref{sec:WinRequired} are installed. Then refer to the appropriate section : If you are more familiar with the graphical user interface, prefer the \textbf{Microsoft visual C++} compiler using the IDE "Visual Studio". Otherwise, if you are friendly with Unix tools and command line terminal, prefer \textbf{MinGW "Minimalist GNU for Windows"} C++ compiler using the command terminal.  




\begin{itemize}
\item Basic installation using \textbf{the command terminal}, refer to Section \ref{sec:WinBasicInstall}.
\item Basic installation using \textbf{the IDE "Visual Studio"}, refer to Section \ref{sec:WinVisualStudioBasicInstall}. 
\end{itemize}


Finally in Section \ref{sec:WinCustomInstall}, the configure options available to build the FA$\mu$ST toolbox are described to propose an Custom - Advanced installation. For example, the optional configuration can be used to modify the install directory path, or to build in debug mode.  



\section{Required components}\label{sec:WinRequired}
The installation of the FA$\mu$ST tools depends on other components to be installed in order to run properly. 
\begin{enumerate}

\item \textbf{Install CMake}: Download Binary distributions correspond to your environment from \url{https://cmake.org/download/}.
\item \textbf{Verify CMake install} : Open a terminal and type the following command:
\begin{lstlisting}
where cmake
\end{lstlisting}
You should obtain the path of your \texttt{cmake.exe} binary file. If not, please add the directory of your \texttt{cmake.exe} file in your environment variable. (to add an environment variable, follow instructions in \ref{sec:ANNEXEEnvironmentVariableWindows}). 



%The directory of binary must be add to the environment PATH of your system if you want to use the cmake command line tool. 
% \item \textbf{Install 7-Zip} from \url{http://www.7-zip.org/}. 7-Zip is a file archiver used to extract external library files. Please verify that \texttt{7z.exe} is present in your environment PATH of your system.

\item \textbf{Install Matlab} (\url{https://fr.mathworks.com/downloads/})

\item \textbf{Verify Matlab install} by typing in a terminal the following command : 
\begin{lstlisting}
which matlab
which mex
\end{lstlisting}
You must obtain the path of your matlab and mex binaries files like: 
\begin{lstlisting}
/usr/local/bin/matlab
/usr/local/bin/mex
\end{lstlisting}
If not, add \texttt{matlab} and \texttt{mex} binaries directories in your environment path (in your ~/.bashrc file). 

\item \textbf{Install Matlab} The builder of the FA$\mu$ST tools automatically checks your Matlab root directory if your \texttt{matlab.exe} is present in your environment Path and/or if your Matlab installation has been performed in a default directory like \texttt{"C:/Program Files/MATLAB/<R2015b>/bin/matlab.exe"} or \texttt{"C:/Program Files (x86)/MATLAB/<R2015b>/bin/matlab.exe"}. In case of several versions of Matlab installed in your system, you can force the directory of your preferred version of Matlab using the following system variable : \\
\texttt{MATLAB\_EXE\_DIR="C:/Program Files/MATLAB/<R2015b>/bin/matlab.exe"}



\textit{Note for the case of using the compiler MinGW :} In Matlab, you must install MinGW version 4.9.2 from MATLAB using the \textbf{ADDON menu}. For more detail, please follow the instruction given in following link :  
\url{http://fr.mathworks.com/help/matlab/matlab_external/install-mingw-support-package.html}. For that, you must have a id session for Mathwork. It is easy to create. 
Current this latest step, an environment variable called MW\_MINGW64\_LOC is automatically generated. 






\item \textbf{Install C++ Compiler:} Both \textbf{Microsoft visual C++} and \textbf{MinGW "Minimalist GNU for Windows"} compiler have been tested. If you are more familiar with the graphical user interface, prefer the \textbf{Microsoft visual C++} compiler. Otherwise, if you are friendly with Unix tools and command line terminal, preferd \textbf{MinGW "Minimalist GNU for Windows"} C++ compiler. The version of the C++ compiler must be coherent with the version of your Matlab version. In this documentation, the version of our C++ compiler corresponds to Matlab 2014 and 2015. If you use an other version of Matlab, please refer to the Mathworks website \url{http://fr.mathworks.com/support/compilers/<R20XXa>}.

\paragraph{}For \textbf{Microsoft visual C++} installation :
\begin{itemize}
\item Download and install Microsoft Visual C++ 2013 professional from \url{https://www.microsoft.com/en-US/download/details.aspx?id=44916}
% \item Download and install Microsoft .NET Framework 4
% \item Download and install Microsoft SDK 7.1
\end{itemize}

\paragraph{}For \textbf{MinGW} installation :
\begin{itemize}
\item Download Mingw in \url{https://sourceforge.net/projects/mingw/files/latest/download?source=files}
\item Launch install file and choose MINGW version 4.9.2 for mexFunction compatibility 
\item The directory of binary must be add to the environment PATH. 

\item \textit{Note for make tool :} In a terminal command, type \texttt{make}. if it doesn't exist, please check if \textbf{make.exe} file is present in MINGW install directory. if not, you can copy and rename mingw32-make.exe to make.exe
\end{itemize}

\end{enumerate}





\section{Basic Build \& Installation}\label{sec:WinBasicInstall}
\paragraph{}When prerequisites listed in previous section \ref{sec:WinRequired} are checked, the FA$\mu$ST installation can be done. 
First download the FA$\mu$ST package on the website  \url{http://faust.gforge.inria.fr/}. 

Depending to your C++ compiler (MinGW or Visual studio), refer to the right part and follow the given instructions.
\\

\section{Basic Build \& Install Using MinGW compiler:}
\label{sec:WinMinGWBasicInstall}

\begin{itemize}
\item Open a command terminal
\item Place you in your local FA$\mu$ST directory (NOTE: don't use any special character in your FAUST directory path, for example the character $\mu$)
\item Type the following commands : 

\begin{lstlisting}
mkdir build
cd build
cmake -G "MinGW Makefiles" .. 
make
make install % run with administrator privilege
\end{lstlisting}
\paragraph{NOTE:} You can generated the CodeBlocks project with the following command : \\
\texttt cmake -G "CodeBlocks - MinGW Makefiles" .. 

\end{itemize}

\section{Basic Build \& Install Using Visual Studio IDE}\label{sec:WinVisualStudioBasicInstall}
\paragraph{}In the case of \textbf{Microsoft Visual Studio 2013 compiler using the Graphical Users Interfaces}:
\begin{enumerate}
\item Open application \texttt{cmake-GUI.exe} from the program menu or from your cmake install binaries directory  to launch the CMake configuration application:

\begin{figure}[!h] %%[!htbp]
\centering
\includegraphics[scale=0.5]{images/cmakeGUI-1-eps-converted-to.pdf}
\caption{cmake GUI}
\label{fig:cmakeGUI-1}
\end{figure}

\item Set the "Where is the source code:" text box with the path of the directory where the source files are located (F:/WORK/FAUST/faust) and the "Where to build the binaries:" with the path of the directory where you want to build the library and executable files (F:/WORK/FAUST/faust/build). (see fig.  \ref{fig:cmakeGUI-1}).

When clicking for the first time on the [Configure] button, CMake will ask for the build tool you want to use. The build system type depends on the builder you want to use, in our case this is the Visual Studio X (X depending the version of Visual installed on the computer) chain tools. (see fig. \ref{fig:cmakeGUI-2}).


\begin{figure}[!h]
\centering
\includegraphics[scale=0.5]{images/cmakeGUI-2-eps-converted-to.pdf}
\caption{cmake GUI}
\label{fig:cmakeGUI-2}
\end{figure}

\item When pressing again the [Configure] button to configure the build system, CMake performs a list of tests to determine the system configuration and manage the build system. If the configuration is correct then no pop-up will appears during the tests and CMake finally shows the various options of the build underlaid in grey. In case of a configuration issue, a pop up window warns you about this issue indicating which test has failed, in this case the build option in the CMake application software will be underlaid in red. We will discuss in Section \ref{sec:WinCustomInstall} what to do in such a case, but let us for the moment assume that everything ran smoothly.
(see \ref{fig:cmakeGUI-4}).

\begin{figure}[!h] %%[!htbp]
\centering
\includegraphics[scale=0.5]{images/cmakeGUI-4-eps-converted-to.pdf}
\caption{cmake GUI}
\label{fig:cmakeGUI-4}
\end{figure}

\item Once the build system configured then generated, you have to actually build FAUST, using Visual Studio.
\item Open file "faust.sln" with visual studio 
\item Click right on Target ALL\_BUILD and select generated 
\item Click right on Target INSTALL and select generated 
\item Click right on Target CTEST and select generated 
\end{enumerate}


\bigbreak
\paragraph{NOTE:}In the case of \textbf{Microsoft Visual Studio 2013 compiler using the command terminal} :

\begin{itemize}
\item Open a command terminal
\item Place you in your local FA$\mu$ST directory (NOTE: don't use any special character in your FAUST directory path, for example the character $\mu$)
\item Type the following commands : 

\begin{lstlisting}
mkdir build
cd build
cmake .. 
cmake --build . --config "Release" --target "install"
\end{lstlisting}

\end{itemize}

\section{Custom - Advanced Installation}\label{sec:WinCustomInstall}

progress... 



\chapter{QuickStart}\label{sec:firstUse}


\paragraph{}A Matlab wrapper is delivered with the FA$\mu$ST C++ library.
It provides a user friendly new class of matrix \textbf{Faust} efficient for the multiplication with matlab built-in dense matrix class.\newline
In order to use matlab wrapper after the installation of Faust, launch Matlab.
In the Matlab terminal, set your working directory to /"HOMEDIR"/Documents/MATLAB/Faust and set the Matlab path by typing the following commands :

\begin{lstlisting}
cd /"HOMEDIR"/Documents/MATLAB/Faust
setup_Faust
\end{lstlisting}

\section{Use a faust from a saved one}\label{sec:firstUseBuildFromSave}
\paragraph{} Now, you can run quick\_start.m script in the Matlab terminal by typing :
\begin{lstlisting}
quick_start
\end{lstlisting}
\paragraph{}In this script, first of all, a Faust of size 4000x5000 is loaded from a previous one that is saved into a matfile :
\lstinputlisting[firstline=47,lastline=48]{../../misc/demo/Quick_start/quick_start.m}
\paragraph{}Secondly, a list of overloaded matlab function shows that a Faust is handled as a normal Matlab builtin matrix.
 
\lstinputlisting[firstline=51,lastline=78]{../../misc/demo/Quick_start/quick_start.m}

\paragraph{}Finally, it performs a little time comparison between multiplication by a Faust or its full matrix equivalent.
This is in order to illustrate the speed-up induced by the Faust. This speed-up should be around 30 (depending on your machine).
\lstinputlisting[firstline=84,lastline=100]{../../misc/demo/Quick_start/quick_start.m}


A \textbf{Faust} object can be constructed from several ways.

\section{Other ways to build a Faust}\label{sec:firstUseBuildFromCellArray}
\paragraph{} A Faust can initialized from a matrix or from a cell-array storing its factors.

\subsection{build a Faust from a matrix and parameters}\label{sec:firstUseBuildFromMatrix}
First, you can build a Faust from a cell-array of matlab matrix (sparse or dense) representing its factors.
\newline
\newline
The following example shows how to build a random faust of size 5x3 with 3 factors :

\begin{lstlisting}
>> nb_factor = 3;
>> dim1 = 5;
>> dim2 = 3; 
>>
>> % list of factors
>> factors = cell(nb_factor);
>>
>> factors{1}=sprand(dim1,dim2,0.1); % first factor is rectangular and sparse 
>> for i=2:nb_factor
>> 		factors{i}=rand(dim2,dim2); % second is sparse
>> end
>>
>> % build the faust
>> A=Faust(factors);
\end{lstlisting}
\newpage

An optional multiplying scalar argument can be taken into account  :
\begin{lstlisting}
>> % multiplicative scalar
>> lambda = 3.5 
>> % build the faust
>> A=Faust(factors,lambda);
\end{lstlisting}

This functionality allows you to build a Faust from the factorization algorithm :
\textbf{mexHierarchical{\_}fact} or \textbf{mexPalm4MSA} :
\begin{lstlisting}
>> % factorization step
>> [lambda,factors]=mexHierarchical_fact(params);
>> % build the faust corresponding to the factorization
>> A=Faust(factors,lambda);
\end{lstlisting}


 

\section{Example}\label{sec:example}

TODO

\chapter{Annexes}\label{sec:Annexes}

\section{Required packages}\label{sec:ANNEXERequiredPackages}
Here is a list of packages used in the FA$\mu$ST project. The installation of this packages are automatically done. There are nothing to do. (see the source directory "./externals").
\begin{itemize}
\item Library \textbf{Eigen} \url{http://eigen.tuxfamily.org}: C++ template library for linear algebra: matrices, vectors, numerical solvers, and related algorithms.
\item Library \textbf{OpenBLAS} \url{http://www.openblas.net}:  Optimized BLAS library based on GotoBLAS2 1.13 BSD version.
\item Library \textbf{xml2} \url{http://xmlsoft.org}
\item Library \textbf{matio} \url{https://sourceforge.net/projects/matio}
\end{itemize}

\section{Compatibility between MATLAB and compiler gcc}\label{sec:ANNEXECompatibilityMatlabCompiler}
Adjust your version of GCC compiler in order to run the installation properly. The use of the mex function in Matlab requires that you have a third-party compiler installed on your system. The latest version of Matlab (2016a in our case) only supports up to GCC 4.7 (see \url{http://fr.mathworks.com/support/compilers/R2016a/index.html?sec=glnxa64} for more detail).
To temporally change your version of gcc compiler, you can modify the environment variable CC an CXX. For that, export your CC and CXX variables corresponding to gcc and g++ binaries path :
\begin{itemize}
\item find your gcc and g++ version path using \texttt{which} command in a terminal :
\begin{lstlisting}
> which gcc
> which g++
\end{lstlisting}

\item Open your ~/.bashrc file and save the return-path of gcc and g++ like:
\begin{lstlisting}[backgroundcolor=\color{white}]
# export version of gcc
export CC=/usr/lib64/ccache/gcc
export CXX=/usr/lib64/ccache/g++
\end{lstlisting}
\end{itemize}

An other way to change the version of compiler use by mex function from matlab command is : 
\begin{lstlisting}
> mex -setup
\end{lstlisting}


\section{Further information about Build \& Install process}\label{sec:ANNEXEInfoBuildInstall}
When using the \textbf{cmake} command to generate the build system, \textbf{cmake} performs a list of tests to determine the system configuration and manage the build system. If the configuration is correct then the build system is generated and written. In this case, the three last lines of the console log of \textbf{cmake} command should be:
\begin{lstlisting}[backgroundcolor=\color{white}]
-- Configuring done 
-- Generating done 
-- Build files have been written to: <YOUR/LOCAL/DIRECTORY/build>
\end{lstlisting}

The command \textbf{make} will compile the build files.\\

The command \textbf{sudo make install} will install the library and others components in the default directory: \\
\texttt{/usr/local/lib/libfaust.a} for the FA$\mu$ST library, \\
\texttt{\textasciitilde /Documents/MATLAB/faust/} for the wrapper matlab.\\
You must have administrator privilege because the library file \texttt{libfaust.a} is copied in an root path directory. If you do not have administrator privilege, you can realize a local install using \texttt{cmake} optional parameter \texttt{-DCMAKE\_INSTALL\_PREFIX="<Your/Install/Dir>"}. 

The \texttt{cmake} optional parameter \texttt{-DCMAKE\_INSTALL\_PREFIX="<Your/Install/Dir>"} allows to install the binaries on the selected install directory. 

The \texttt{cmake} optional parameter \texttt{-G "CodeBlocks - Unix Makefiles"} allows to generate the Code Blocks project and the Unix Makefiles.\\ 
The \texttt{cmake} optional parameter \texttt{-G "Xcode"} allows to generate the Xcode project. 




%for windows
\section{Required packages on Windows platform}\label{sec:WinRequiredPackages}
Here is a list of packages used in the FA$\mu$ST project. Eigen and OpenBlas library are automatically installed : there are nothing to do (see the directory "./externals/win/").
\begin{itemize}
\item \textbf{Eigen} is a C++ template library for linear algebra: matrices, vectors, numerical solvers, and related algorithms (see \url{http://eigen.tuxfamily.org}).
\item \textbf{OpenBLAS} is an optimized BLAS library based on GotoBLAS2 1.13 BSD version. (see \url{http://www.openblas.net}). To install OpenBlas, refer to \url{https://github.com/xianyi/OpenBLAS/wiki/Installation-Guide}. You can directly download precompiled binary here \url{https://sourceforge.net/projects/openblas/files/v0.2.14/}
\end{itemize}



\section{Add environment variable on Windows platform}\label{sec:ANNEXEEnvironmentVariableWindows}
Here is the steps to add an environment variable in Windows 7.
\begin{enumerate}
\item From the Desktop, right-click the Computer icon and select Properties. If you don't have a Computer icon on your desktop, click the Start button, right-click the Computer option in the Start menu, and select Properties.
\item Click the Advanced System Settings link in the left column.
\item In the System Properties window, click on the Advanced tab, then click the Environment Variables button near the bottom of that tab.
\item In the Environment Variables window (pictured below), highlight the Path variable in the "System variables" section and click the Edit button. Add or modify the path lines with the paths you want the computer to access. Each different directory is separated with a semicolon as shown below.
\begin{lstlisting}[backgroundcolor=\color{white}]
C:\Program Files;C:\Winnt;C:\Winnt\System32
\end{lstlisting}

\begin{figure}[!h] %%[!htbp]
\centering
\includegraphics[scale=0.5]{images/EnvironmentVariable.jpeg}
%\caption{cmake GUI}
\label{fig:EnvironmentVariable}
\end{figure}

\end{enumerate}





\bibliographystyle{plain}
\bibliography{paperbiblio}


\end{document}



