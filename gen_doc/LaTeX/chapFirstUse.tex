\chapter{QuickStart}\label{sec:firstUse}


\paragraph{}A Matlab wrapper is delivered with the FA$\mu$ST C++ library.
It provides a user friendly new class of matrix \textbf{Faust} efficient for the multiplication with matlab built-in dense matrix class.\newline
In order to use matlab wrapper after the installation of Faust, launch Matlab.
In the Matlab terminal, set your working directory to /"HOMEDIR"/Documents/MATLAB/Faust and set the Matlab path by typing the following commands :

\begin{lstlisting}
>> cd /"HOMEDIR"/Documents/MATLAB/Faust
>> setup_Faust
\end{lstlisting}

\section{Use a faust from a saved one}\label{sec:firstUseBuildFromSave}
\paragraph{} Now, you can run quick\_start.m script in the Matlab terminal by typing :
\begin{lstlisting}
>> quick_start
\end{lstlisting}
\paragraph{}In this script, first of all, a Faust of size 4000x5000 is loaded from a previous one that is saved into a matfile :
\lstinputlisting[firstline=47,lastline=48]{../../misc/demo/Quick_start/quick_start.m}
\paragraph{}Secondly, a list of overloaded matlab function shows that a Faust is handled as a normal Matlab builtin matrix.
 
\lstinputlisting[firstline=51,lastline=78]{../../misc/demo/Quick_start/quick_start.m}

\paragraph{}Finally, it performs a little time comparison between multiplication by a Faust or its full matrix equivalent.
This is in order to illustrate the speed-up induced by the Faust. This speed-up should be around 30 (depending on your machine).
\lstinputlisting[firstline=84,lastline=100]{../../misc/demo/Quick_start/quick_start.m}


A \textbf{Faust} object can be constructed from several ways.

\section{Other ways to build a Faust}\label{sec:firstUseBuildFromCellArray}
\paragraph{} A Faust can initialized from a matrix or from a cell-array storing its factors.

\subsection{build a Faust from a matrix and parameters}\label{sec:firstUseBuildFromMatrix}
First, you can build a Faust from a cell-array of matlab matrix (sparse or dense) representing its factors.
\newline
\newline
The following example shows how to build a random faust of size 5x3 with 3 factors :

\begin{lstlisting}
>> nb_factor = 3;
>> dim1 = 5;
>> dim2 = 3; 
>>
>> % list of factors
>> factors = cell(nb_factor);
>>
>> factors{1}=sprand(dim1,dim2,0.1); % first factor is rectangular and sparse 
>> for i=2:nb_factor
>> 		factors{i}=rand(dim2,dim2); % second is sparse
>> end
>>
>> % build the faust
>> A=Faust(factors);
\end{lstlisting}
\newpage

An optional multiplying scalar argument can be taken into account  :
\begin{lstlisting}
>> % multiplicative scalar
>> lambda = 3.5 
>> % build the faust
>> A=Faust(factors,lambda);
\end{lstlisting}

This functionality allows you to build a Faust from the factorization algorithm :
\textbf{mexHierarchical{\_}fact} or \textbf{mexPalm4MSA} :
\begin{lstlisting}
>> % factorization step
>> [lambda,factors]=mexHierarchical_fact(params);
>> % build the faust corresponding to the factorization
>> A=Faust(factors,lambda);
\end{lstlisting}


 
