\chapter{QuickStart}\label{sec:firstUse}


\paragraph{}A Matlab wrapper is delivered with the FA$\mu$ST C++ library.
It provides a user friendly new class of matrix \textbf{Faust} efficient for the multiplication with matlab built-in dense matrix class.\newline

\section{Configure Matlab path}\label{sec:firstUseMatlabPath}
In order to use matlab wrapper, follow the instructions :
\begin{itemize}
\item \textbf{Install} FA$\mu$ST tool (see preceding chapter)
\item \textbf{Launch} Matlab.
\item \textbf{Set the working directory} of the Matlab Command Window to :\\
\texttt{/<HOMEDIR>/Documents/MATLAB/Faust}
\item \textbf{Configure} the Matlab path by typing the following commands :
\begin{lstlisting}
>> cd /<HOMEDIR>/Documents/MATLAB/Faust
>> setup_Faust
\end{lstlisting}

\end{itemize}

\section{Use a faust from a saved one}\label{sec:firstUseBuildFromSave}
\paragraph{} Now, you can run \texttt{quick\_start.m} script in the Matlab Command Window by typing :
\begin{lstlisting}
>> quick_start
\end{lstlisting}
\texttt{quick\_start.m} script is located in following path :\\
\texttt{<HOMEDIR>/Documents/MATLAB/faust/demo/Quick\_start/quick\_start.m} \\
In this script, first of all, a Faust of size 4000x5000 is loaded from a previous one that is saved into a matfile located in :\\
\texttt{<HOMEDIR>/Documents/MATLAB/faust/demo/Quick\_start/faust\_quick\_start.mat}
\lstinputlisting[firstline=47,lastline=48,backgroundcolor=\color{white}]{../../misc/demo/Quick_start/quick_start.m}
\newpage
\paragraph{}Secondly, a list of overloaded matlab function shows that a Faust is handled as a normal Matlab builtin matrix.
 
\lstinputlisting[firstline=51,lastline=78,backgroundcolor=\color{white}]{../../misc/demo/Quick_start/quick_start.m}

\paragraph{}Finally, it performs a little time comparison between multiplication by a Faust or its full matrix equivalent.
This is in order to illustrate the speed-up induced by the Faust. This speed-up should be around 30 (depending on your machine).
%\lstinputlisting[firstline=84,lastline=100,backgroundcolor=\color{white}]{../../misc/demo/Quick_start/quick_start.m}

\newpage
\section{Construct a Faust from a given matrix}\label{sec:firstUseBuildFromMatrix}
\paragraph{} To see an example of building a Faust from a matrix, you can run \texttt{factorise\_matrix.m} in the Matlab Command Window by typing :
\begin{lstlisting}
>> factorise_matrix
\end{lstlisting}
\texttt{factorise\_matrix.m} script is located in following path :\\
\texttt{<HOMEDIR>/Documents/MATLAB/faust/demo/Quick\_start/factorise\_matrix.m} \\

In this script, from a given matrix A of size 100x200 :
\lstinputlisting[firstline=42,lastline=47,backgroundcolor=\color{white}]{../../misc/demo/Quick_start/factorise_matrix.m}
We generate the parameters of the factorisation from :
\begin{itemize}
\item The dimension of A (\textbf{dim1} and \textbf{dim2}),
\item \textbf{nb\_factor} the number of factor of the Faust,
\item \textbf{rcg} the Rational Complexity Gain, which represents the theoretical memory gain and multiplication speed-up of the Faust compared to the initial matrix 
\end{itemize}

\lstinputlisting[firstline=51,lastline=56,backgroundcolor=\color{white}]{../../misc/demo/Quick_start/factorise_matrix.m}
Then we factorize the matrix \textbf{A} into a Faust \textbf{Faust\_A}
\lstinputlisting[firstline=58,lastline=59,backgroundcolor=\color{white}]{../../misc/demo/Quick_start/factorise_matrix.m}
And as for quickstart.m, we make some time comparison at the end.

\newpage
\section{Construct a Faust from its factor}\label{sec:firstUseBuildFactors}
To see an example of building a Faust from its factors, you can run construct\_Faust\_from\_factors.m in the Matlab Command Window by typing :
\begin{lstlisting}
>> construct_Faust_from_factors
\end{lstlisting}
This following example shows how to build a faust from a cell-array representing its factors.
\lstinputlisting[firstline=44,lastline=72,backgroundcolor=\color{white}]{../../misc/demo/Quick_start/construct_Faust_from_factors.m}


 
