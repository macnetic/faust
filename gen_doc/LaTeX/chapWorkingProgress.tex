%!TEX root =  gettingStartedFAuST-version2_0.tex
\chapter{Work in Progress}\label{sec:WorkingProgress}
	Various features are still under development and not yet stable.
But here is an overview of a roadmap for features that we plan to improve/integrate in an upcoming release :
\begin{itemize}
	\item \textbf{Factorization algorithms :} (cf. Section \ref{sec:WorkingProgressBuildFromMatrix})\\ 
A beta version is available in this release, implementing in C++ what what delivered in plain Matlab in version 1.0.0. 
Refer to Section \ref{sec:WorkingProgressBuildFromMatrix}) for more information on how to play with these algorithms, which C++ implementation and API are not yet stabilized.
	\item \textbf{Graphics Processing Unit (GPU) :} (cf. Section \ref{sec:OptionalGPU})\\
A GPU version of the code is under development, it already shows 
some time savings compared to the CPU code for certain computations, but is not enough user-friendly and not easy to install to be in this release. The most advanced development is currently on Linux, you can play with it following the steps described in Section \ref{sec:OptionalGPU}. Report any problem and/or suggestion to the mailing list \url{http://lists.gforge.inria.fr/pipermail/faust-install/}. 
	\item \textbf{Python wrapper :} A Python wrapper is also planned, it will used the Cython module.
	\item \textbf{Command line wrapper :} A command line wrapper is also envisioned. It will integrated the externals Libraries "XML" and "MATIO" to manipulate the data (configuration and matrix). 
	\item \textbf{A||GO wrapper :} A wrapper for the A||GO platform of web services is also planned to propose a demonstration on the website \url{https://allgo.inria.fr/}.
		
	\item \textbf{Image denoising experiment :} (cf \cite[chapter VI]{LeMagoarou2016}) \\
This experiment was in the previous release (FA$\mu$ST version 1.0, pure Matlab implementation) but is not in the current one because our C++ wrapper is not yet compatible with the sparse decomposition algorithm used in this experiment. Please use version 1.0.0 to reproduce this image denoising experiment. 
\end{itemize}

\section{Python wrapper : beta version}\label{sec:Pythonwrapper}
This section describes the different step to install the Python wrapper and to get acustomed to its behaviour.
The Python wrapper has only been tested on Unix system, not Windows.

Throughout this section we provide examples as follows:
\begin{itemize}
\item Command lines you must enter in a terminal are displayed as:
\lstset{style=customBash}
\begin{lstlisting}
> mkdir BUILD; ls . 
\end{lstlisting}
\item Messages resulting from such command lines appear without the leading '>':
\lstset{style=customBash}
\begin{lstlisting}
example of return text in your current terminal. 
\end{lstlisting}
\item Python code is displayed like this :
\lstset{style=customPython}
\begin{lstlisting}
# this is a commentary
y = A*x; 
\end{lstlisting}
\end{itemize}

\subsection{Required Components}
This Section lists the required components you must install before to begin the FA$\mu$ST installation.

\begin{itemize}
\item \textbf{Install Python}. The wrapper has only been tested with Python version 2 (2.7.x) (\url{https://cmake.org/download/}).
To install Python 2, type the following command :
\lstset{style=customBash}
\begin{lstlisting}
> pip install python2
\end{lstlisting}
But Python 2 is a native package on many Unix system.
 
\item \textbf{Install Numpy}
Numpy is a Python package useful linear algebra (http://www.numpy.org/)
It is often delivered with Python but in case you haven't it,you can install it by typing the following command :
\lstset{style=customBash}
\begin{lstlisting}
> pip install numpy
\end{lstlisting}

Check the version of the Numpy package,
by typing the following command :
\lstset{style=customBash}
\begin{lstlisting}
> pip freeze | grep 'numpy'
\end{lstlisting}

\textbf{Warning} : version of Numpy older than 1.9.2 may be not compatible with the Python wrapper
 
\item \textbf{Install Cython}
Cython is a Python package, that allows to interface C/C++ with Python. (\url{http://cython.org/})

It is often delivered with Python but in case you haven't it,you can install it by typing the following command :
\lstset{style=customBash}
\begin{lstlisting}
> pip install cython
\end{lstlisting}

\end{itemize}



\subsection{Install}
The installation is very similar to the installation of the Matlab wrapper. 


\begin{itemize}
\item \textbf{Type the following commands :} 
\lstset{style=customBash}
\begin{lstlisting}
> mkdir build
> cd build
> cmake .. -DBUILD_WRAPPER_PYTHON=ON 
> make
> sudo make install # run with administrator privilege
\end{lstlisting}


\item By default,
The Python wrapper is installed in the directory : \texttt{<HOMEDIR>\textbackslash Documents\textbackslash PYTHON\textbackslash faust}.

But you can choose your own install directory by setting the \newline
CMAKE variable  \textbf{CMAKE\_INSTALL\_PYTHON\_PREFIX}.



Type the following command :
 \lstset{style=customBash}
\begin{lstlisting}
> mkdir build
> cd build
> cmake .. -DBUILD_WRAPPER_PYTHON=ON -DCMAKE_INTALL_PYTHON_PREFIX="YourPath" 
> make
> sudo make install # run with administrator privilege
\end{lstlisting}

\end{itemize}



\subsection{Quickstart}
	This subsection presents the quickstart demo to get acustomed to the Python wrapper.
	 Now, you can go to the install directory of the Python wrapper which is \newline \texttt{<HOMEDIR>\textbackslash Documents\textbackslash PYTHON\textbackslash faust} by default.
	\newline
	
	\begin{itemize}	
	\item In a terminal, type the following command :
	 \lstset{style=customBash}
	\begin{lstlisting}
	> cd ~/Documents/PYTHON/faust/
	\end{lstlisting}
	\item You can run the Python script \texttt{quickstart.py}
	by typing the following command :
	\lstset{style=customBash}
	\begin{lstlisting}
	> python quickstart.py
	\end{lstlisting}
	\end{itemize}
	
	In this script
	\begin{enumerate}
		\item We import the FaustPy package
		\lstinputlisting[firstline=39,lastline=40,style=customPython]{../../wrapper/python/quickstart.py}
		\item We create a Faust from a list of factors represented as Numpy matrices
		\lstinputlisting[firstline=64,lastline=65,style=customPython]{../../wrapper/python/quickstart.py}
		\item A list of overloaded Numpy operation shows that a Faust is handled as a normal
			  Numpy matrix.
		\lstinputlisting[firstline=67,lastline=83,style=customPython]{../../wrapper/python/quickstart.py}
		\item It performs a little time comparison between multiplication by a FA$\mu$ST or its Numpy equivalent matrix.
		\lstinputlisting[firstline=67,lastline=83,style=customPython]{../../wrapper/python/quickstart.py}	  
	\end{enumerate}
		
	
\section{Construct a FAuST from a given matrix}\label{sec:WorkingProgressBuildFromMatrix}
Please ensure that you have configured your Matlab environment (cf. Section \ref{sec:firstUseMatlabPath}). Then, to see an example of building a FA$\mu$ST from a matrix, you can run \texttt{factorize\_matrix.m} in the Matlab Command Window by typing :
\lstset{style=customMatlab}
\begin{lstlisting}
>> factorize_matrix
\end{lstlisting}
\texttt{factorize\_matrix.m} script is located in the following path :\\
\texttt{<FAuST\_INSTALL\_DIR>/demo/Quick\_start/factorize\_matrix.m} \\

In this script, from a given matrix A of size 100x200 
\lstinputlisting[firstline=42,lastline=47,style=customMatlab]{../../misc/demo/Quick_start/factorize_matrix.m}
we generate the parameters of the factorization from :
\begin{itemize}
\item The dimension of A (\textbf{dim1} and \textbf{dim2}),
\item \textbf{nb\_factor:} the desired number of factors of the FA$\mu$ST,
\item \textbf{rcg:} the targeted {\em Relative Complexity Gain}, which represents the theoretical memory gain and multiplication speed-up of the FA$\mu$ST compared to the initial matrix .
\\ \\\textbf{WARNING :}  A trade-off exists between the targeted RCG/speed-up of the FA$\mu$ST and the achievable data fidelity to the input matrix. The higher the RCG, the higher the error of the FA$\mu$ST relative to the input matrix.
\end{itemize}

\lstset{style=customBash}
\lstinputlisting[firstline=51,lastline=56,style=customMatlab]{../../misc/demo/Quick_start/factorize_matrix.m}
Then we factorize the matrix \textbf{A} into a FA$\mu$ST \textbf{Faust\_A}
\lstinputlisting[firstline=58,lastline=59,style=customMatlab]{../../misc/demo/Quick_start/factorize_matrix.m}


\section{Installing FAuST to enable GPU acceleration}\label{sec:OptionalGPU}
As a beta version, the FA$\mu$ST toolbox integrates optional GPU (Graphics Processing Unit) acceleration to improve its time performance.
\paragraph{Warning:} Currently, this optional GPU install has only be implemented and tested on a Linux machine. There is no guarantee that the installation and the use will be effective for every system.

\begin{itemize}
\item \textbf{Install} the CUDA Toolkit from NVIDIA website:\\
\url{https://developer.nvidia.com/cuda-downloads}).
\item \textbf{Install} the drivers for NVIDIA from NVIDIA website:\\ \url{http://www.nvidia.fr/Download/index.aspx}.
\item \textbf{Verify install} of GPU tools by typing in a terminal :
\lstset{style=customBash} 
\begin{lstlisting}
> which nvcc
\end{lstlisting}
You must obtain the path of your \texttt{nvcc} compiler like 
\begin{lstlisting}
/usr/local/cuda-7.5/bin/nvcc
\end{lstlisting}
If not, add \texttt{nvcc} directory in your environment path (in your ~/.bashrc file).

\item \textbf{Verify install of GPU library} by typing in a terminal: 
\begin{lstlisting}
> echo CUDADIR
\end{lstlisting}
You must obtain the path of your cuda directory like
\begin{lstlisting}
/usr/local/cuda-7.5
\end{lstlisting}
If not, export CUDADIR (in your .bashrc file for example).
\begin{lstlisting}
export CUDADIR=/usr/local/cuda-7.5
\end{lstlisting}

		
	   
\end{itemize}

When prerequisities listed in Section \ref{sec:RequiredTools} are checked, you can get the package FA$\mu$ST.
\begin{itemize}
\item \textbf{Download} the FA$\mu$ST package on the website :  \url{http://faust.gforge.inria.fr/}
\item \textbf{Unzip} the FA$\mu$ST package into your FA$\mu$ST directory.
\item \textbf{Open} a command terminal
\item \textbf{Set the current directory} to your FA$\mu$ST directory (NOTE: do not use any special character in your FA$\mu$ST directory path, for example the character $\mu$) and type :

\lstset{style=customBash}
\begin{lstlisting}
> mkdir build
> cd build
> cmake -DBUILD_USE_GPU="ON" ..
> make
> sudo make install # run with administrator privilege
\end{lstlisting}

\end{itemize}


The FA$\mu$ST Toolbox should be installed. Now, refer to Quick-Start Chapter \ref{sec:firstUse} to check the install and to try FA$\mu$ST toolbox \textbf{using GPU process}.

