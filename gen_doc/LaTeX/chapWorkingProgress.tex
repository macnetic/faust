\chapter{Working Progress}\label{sec:WorkingProgress}
Various features are still under development and not yet stable.
But here is a list of features that will be improved/integrated in the next release :
\begin{itemize}
	\item \textbf{Factorization algorithm : } they are available in this release. You can even used them
from user-friendly approach (cf Section \ref{sec:WorkingProgressBuildFromMatrix}) but they are not
that stable. In the sense that they are not fitted to all types of matrices.
	\item \textbf{GPU : } A GPU version of the code is under development, it already shows 
some time saving compared to the CPU code but is not enought user-friendly, easy to install to be in this release.
	\item \textbf{Python wrapper : } A Python wrapper is also planned, it will used Cython module.
	\item \textbf{Image denoising experiment (cf \cite[chapter VI.]{LeMagoarou2016}) :} This experiment was in the previous release (Faust version 1.0) 
	but is not in the current one because our C++ wrapper is not yet fitted to the algorithm used in this experiment.    
\end{itemize}

\section{Construct a Faust from a given matrix}\label{sec:WorkingProgressBuildFromMatrix}
\paragraph{}
	After checking that you have configured your Matlab environment (cf section \ref{sec:firstUseMatlabPath})  
	To see an example of building a Faust from a matrix, you can run \texttt{factorise\_matrix.m} in the Matlab Command Window by typing :
\lstset{style=customMatlab}
\begin{lstlisting}
>> factorise_matrix
\end{lstlisting}
\texttt{factorise\_matrix.m} script is located in following path :\\
\texttt{<FAuST\_INSTALL\_DIR>/demo/Quick\_start/factorise\_matrix.m} \\

In this script, from a given matrix A of size 100x200 :
\lstinputlisting[firstline=42,lastline=47,style=customMatlab]{../../misc/demo/Quick_start/factorise_matrix.m}
We generate the parameters of the factorisation from :
\begin{itemize}
\item The dimension of A (\textbf{dim1} and \textbf{dim2}),
\item \textbf{nb\_factor} the number of factor of the Faust,
\item \textbf{rcg} the Rational Complexity Gain, which represents the theoretical memory gain and multiplication speed-up of the Faust compared to the initial matrix .
\\ \\\textbf{WARNING :}  A trade-off exists between the RCG/speed-up of the Faust and the data fidelity to the input matrix.
The higher the RCG, the higher the error of the Faust relative to the input matrix
\end{itemize}

\lstset{style=customBash}
\lstinputlisting[firstline=51,lastline=56,style=customMatlab]{../../misc/demo/Quick_start/factorise_matrix.m}
Then we factorize the matrix \textbf{A} into a Faust \textbf{Faust\_A}
\lstinputlisting[firstline=58,lastline=59,style=customMatlab]{../../misc/demo/Quick_start/factorise_matrix.m}

