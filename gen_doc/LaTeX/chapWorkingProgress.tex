\chapter{Working Progress}\label{sec:WorkingProgress}
	Various features are still under development and not yet stable.
But here is a list of features that will be improved/integrated in the next release :
\begin{itemize}
	\item \textbf{Factorization algorithm :} (cf. Section \ref{sec:WorkingProgressBuildFromMatrix})\\ 
They are available in this release. You can even used them
from user-friendly approach (cf. Section \ref{sec:WorkingProgressBuildFromMatrix}) but they are not
that stable. In the sense that they are not fitted to all types of matrices.
	\item \textbf{Graphics Processing Unit (GPU) :} (cf. Section \ref{sec:OptionalGPU})\\
A GPU version of the code is under development, it already shows 
some time saving compared to the CPU code but is not enough user-friendly and not easy to install to be in this release. However, you can try the installation described in Section \ref{sec:OptionalGPU}, and report any problem and/or suggestion on install-list of FA$\mu$ST project on \url{http://lists.gforge.inria.fr/pipermail/faust-install/}. 
	\item \textbf{Python wrapper :} A Python wrapper is also planned, it will used Cython module.
	\item \textbf{Command line wrapper :} A command line wrapper is also planned. It will integrated the externals Libraries "XML" and "MATIO" to manipulate the data (configuration and matrix). 
	\item \textbf{Image denoising experiment :} (cf. \cite[chapter VI.]{LeMagoarou2016}) \\
This experiment was in the previous release (FA$\mu$ST version 1.0) but is not in the current one because our C++ wrapper is not yet fitted to the algorithm used in this experiment.    
\end{itemize}

\section{Construct a FAuST from a given matrix}\label{sec:WorkingProgressBuildFromMatrix}
Please ensure that you have configured your Matlab environment (cf. Section \ref{sec:firstUseMatlabPath}). Then, to see an example of building a FA$\mu$ST from a matrix, you can run \texttt{factorise\_matrix.m} in the Matlab Command Window by typing :
\lstset{style=customMatlab}
\begin{lstlisting}
>> factorise_matrix
\end{lstlisting}
\texttt{factorise\_matrix.m} script is located in following path :\\
\texttt{<FAuST\_INSTALL\_DIR>/demo/Quick\_start/factorise\_matrix.m} \\

In this script, from a given matrix A of size 100x200 :
\lstinputlisting[firstline=42,lastline=47,style=customMatlab]{../../misc/demo/Quick_start/factorise_matrix.m}
We generate the parameters of the factorization from :
\begin{itemize}
\item The dimension of A (\textbf{dim1} and \textbf{dim2}),
\item \textbf{nb\_factor} the number of factor of the FA$\mu$ST,
\item \textbf{rcg} the Rational Complexity Gain, which represents the theoretical memory gain and multiplication speed-up of the FA$\mu$ST compared to the initial matrix .
\\ \\\textbf{WARNING :}  A trade-off exists between the RCG/speed-up of the FA$\mu$ST and the data fidelity to the input matrix.
The higher the RCG, the higher the error of the FA$\mu$ST relative to the input matrix
\end{itemize}

\lstset{style=customBash}
\lstinputlisting[firstline=51,lastline=56,style=customMatlab]{../../misc/demo/Quick_start/factorise_matrix.m}
Then we factorize the matrix \textbf{A} into a FA$\mu$ST \textbf{Faust\_A}
\lstinputlisting[firstline=58,lastline=59,style=customMatlab]{../../misc/demo/Quick_start/factorise_matrix.m}


\section{FAuST installation using Optional GPU process}\label{sec:OptionalGPU}
The toolbox FA$\mu$ST integrates optional Graphics Processing Unit (GPU) to improve its time performances.
\paragraph{Warning:} This optional GPU install has only be realized on a Linux machine. There is no guarantee that the installation and the use will be effective for every system.

\begin{itemize}
\item \textbf{Install} the CUDA Toolkit from NVIDIA website:\\
\url{https://developer.nvidia.com/cuda-downloads}).
\item \textbf{Install} the drivers for NVIDIA from NVIDIA website:\\ \url{http://www.nvidia.fr/Download/index.aspx}.
\item \textbf{Verify install} of GPU tools by typing in a terminal :
\lstset{style=customBash} 
\begin{lstlisting}
> which nvcc
\end{lstlisting}
You must obtain the path of your \texttt{nvcc} compiler like 
\begin{lstlisting}
/usr/local/cuda-7.5/bin/nvcc
\end{lstlisting}
If not, add \texttt{nvcc} directory in your environment path (in your ~/.bashrc file). 
\end{itemize}

When prerequisities listed in Section \ref{sec:RequiredTools} are checked, you can get the package FA$\mu$ST.
\begin{itemize}
\item \textbf{Download} the FA$\mu$ST package on the website :  \url{http://faust.gforge.inria.fr/}
\item \textbf{Unzip} the FA$\mu$ST package into your FA$\mu$ST directory.
\item \textbf{Open} a command terminal
\item \textbf{Set the current directory} to your FA$\mu$ST directory (NOTE: do not use any special character in your FA$\mu$ST directory path, for example the character $\mu$) and type :

\lstset{style=customBash}
\begin{lstlisting}
> mkdir build
> cd build
> cmake -DBUILD_USE_GPU="ON" ..
> make
> sudo make install # run with administrator privilege
\end{lstlisting}

\end{itemize}


FA$\mu$ST Toolbox should be installed. Now, refer to Quick-Start Chapter \ref{sec:firstUse} to check the install and to try FA$\mu$ST toolbox \textbf{using GPU process}.

