\chapter{Installation}\label{sec:install}

The FA$\mu$ST project is based on an C++ library available for both UNIX and Windows environments. CMake has been choose to build the project FA$\mu$ST because it is an open-source, cross-platform family of tools designed to build, test and package software (cf. website \url{https://cmake.org/}).

First section \ref{sec:UnixInstall} explains how to install the project FA$\mu$ST for UNIX platform and second section \ref{sec:WinInstall} corresponds to the Windows installation. 

\section{Unix platform}\label{sec:UnixInstall}

\subsection{Prerequisities for installation}\label{sec:UnixPrerequisitiesInstall}
  

\paragraph{}The use of the mex function in Matlab requires that you have a third-party compiler installed on your system. The latest version of Matlab (2016a in our case) only supports up to GCC 4.7 (see \url{http://fr.mathworks.com/support/compilers/R2016a/index.html?sec=glnxa64} for more detail). Please adjust your version of GCC compiler in order to run the installation properly. 

\paragraph{}[OPTIONAL] The use of GPU process in FAUST project required the drivers for NVIDIA and CUDA install.


\paragraph{}In your environment PATH, please add following components :
\begin{itemize}
\item matlab (example: "export PATH=/usr/local/bin:\$PATH")
\item [OPTIONAL] nvcc for GPU compiler (example: "export PATH=/usr/local/cuda-X.X/bin:\$PATH"
\end{itemize}

\paragraph{}Please export following variable:
\begin{itemize}
\item CC with gcc (example: "export CC='/usr/lib64/ccache/gcc'") 
\item CXX with g++ (example: "export CXX=/usr/lib64/ccache/g++
\end{itemize}


\subsection{FAUST installation}\label{sec:UnixFaustInstall}
When prerequisities listed in precedent section \ref{sec:UnixPrerequisitiesInstall} are checked, the FA$\mu$ST installation can be done : 

\begin{itemize}
\item Download the FA$\mu$ST package on the website :  \url{http://faust.gforge.inria.fr/}
\item Open a command terminal
\item Place you in the FA$\mu$ST directory, and type the following commands : 
\begin{lstlisting}
mkdir build
cd build
cmake ..
make
make install
\end{lstlisting}
\end{itemize}

\subsection{Optional - Advanced Installer}\label{sec:UnixOptionalInstall}

The project FAUST can be configured with optional parameters. They can be activated in the principal CMakeList.txt file, or directly using the cmake command line

$cmake\ ..\ -D<BUILD_NAME>=<ON/OFF>$

\begin{itemize}
\item BUILD\_TESTING : Enable the ctest option (default value is ON)
\item BUILD\_DOCUMENTATION : Generating the doxygen documentation (default value is OFF)  
\item BUILD\_MULTITHREAD : Enable multithread with OpenMP Multithreading (default value is OFF)
\item BUILD\_VERBOSE : Enable verbose option when compile (-v) (default value is OFF)
\item BUILD\_DEBUG : Enable FAUST Debug mode (default value is OFF )
\item BUILD\_USE\_GPU : Using both CPU and GPU process ( default value is OFF)
\item BUILD\_MATLAB\_MEX\_FILES : Enable building Matlab MEX files (default value is ON)
\item BUILD\_OPENBLAS : Using openBLAS for matrix and vector computations (default value is OFF )
\item BUILD\_READ\_XML\_FILE : Using xml2 library to read xml files (default value is OFF)
\item BUILD\_READ\_MAT\_FILE : Using matio library to read mat files (default value is OFF)
\end{itemize}

Following the selected option, the cmake installer automatically checks the dependent component (library OpenBlas, eigen, matio, libxml2).  


\section{Windows platform}\label{sec:WinInstall}

\subsection{Prerequisities for installation}\label{sec:WinPrerequisitiesInstall}
The installation of the FA$\mu$ST project depends on other components to be installed in order to run properly. 
\begin{enumerate}

\item \textbf{Install CMake} for building the FAUST project. 
In \url{https://cmake.org/download/}, download Binary distributions correspond to your environment (in our case  cmake-3.6.1-win64-x64.zip)
The directory of binary must be add to the environment PATH. 

\item \textbf{Install 7-Zip} \url{http://www.7-zip.org/} . 7-Zip is a file archiver used for external library

\item \textbf{Install Matlab} if not already done (MATLAB R2015b in our case).

Note for the case of the use of compiler MinGW : In Matlab, you must install MinGW version 4.9.2 from MATLAB using the \textbf{ADDON menu}. For more detail, please follow the instruction given in following link :  
\url{http://fr.mathworks.com/help/matlab/matlab_external/install-mingw-support-package.html}. For that, you must have a id session for Mathwork. It is easy to create. 
Current this latest step, an environment variable called MW\_MINGW64\_LOC is automatically generated. 


\item \textbf{Install C++ Compiler:} Both \textbf{Microsoft visual C++} and \textbf{MinGW "Minimalist GNU for Windows"} compiler have been tested. The version of this compilers must be coherent with the version of your Matlab version. Here is the compiler installation corresponding to Matlab 2014 and 2015. If you use an other version of Matlab, please refers to the Mathworks website \url{http://fr.mathworks.com/support/compilers/<R20XXa>}.

\paragraph{}For \textbf{Microsoft visual C++} installation :
\begin{itemize}
\item Download and install Microsoft .NET Framework 4
\item Download and install Microsoft SDK 7.1
\item Download and install Microsoft Visual C++ 2013 professional
\end{itemize}

\paragraph{}For \textbf{MinGW} installation :
\begin{itemize}
\item Download Mingw in \url{https://sourceforge.net/projects/mingw/files/latest/download?source=files}
\item Launch install file and choose MINGW version 4.9.2 for mexFunction compatibility 
\item The directory of binary must be add to the environment PATH. 

\item Note for make tool : In a terminal command, type "make". if it doesn't exist, please check if make.exe is present in MINGW install directory. if not, you can copy and rename mingw32-make.exe to make.exe
\end{itemize}



\item In your environment PATH, please verify and add following components :

\begin{itemize}
\item matlab.exe (example: "C:$\setminus$Program Files$\setminus$MATLAB$\setminus$R2015b$\setminus$bin")
\item 7z.exe (example: "C:$\setminus$prog$\setminus$7-Zip")
\item cmake.exe (example: "C:$\setminus$Users$\setminus$ci$\setminus$Documents$\setminus$library$\setminus$cmake-3.6.0-win64-x64$\setminus$bin")
\item In case of the use of MinGW compiler : gcc.exe (example: "C:$\setminus$mingw-w64$\setminus$mingw64$\setminus$bin")
\end{itemize}
\end{enumerate}


\subsection{FAUST installation}\label{sec:WinFaustInstall}
When prerequisities listed in precedent section \ref{sec:WinPrerequisitiesInstall} are checked, the FA$\mu$ST installation can be done. 
\begin{itemize}
\item Download the FA$\mu$ST package on the website :  \url{http://faust.gforge.inria.fr/}
\item Open a command terminal
\item Place you in the FAUST directory, and type the following commands : 

In the case of MinGW compiler :
\begin{lstlisting}
mkdir build
cd build
cmake -G "MinGW Makefiles" .. 
make
make install
\end{lstlisting}

In the case of Microsoft Visual Studio 2013 compiler :
\begin{lstlisting}
mkdir build
cd build
cmake -G "Visual Studio 12 2013" .. 
cmake --build . --config "Release" --target "install"
\end{lstlisting}

\end{itemize}


\subsection{Optional - Advanced Installer}\label{sec:WinOptionalInstall}

progress... 



