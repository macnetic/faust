\chapter{Installation on Windows platform}\label{sec:WinInstall}


The FA$\mu$ST project is based on \textbf{C++ library} available for Linux, MAC OS X and Windows platforms. The proposed toolbox provides a Matlab wrapper. \textbf{CMake} has been chosen to build the FA$\mu$ST project because it is an open-source, cross-platform family of tools designed to build, test and package software. This chapter presents the steps to install the FA$\mu$ST tools on Windows platform.

\begin{itemize}
\item Firstly, please ensure that the \textbf{prerequisite components} listed in Section \ref{sec:WinRequired} are installed. 

\item Secondly, \textbf{choose your preferred C++ Compiler} between GCC from MinGW "Minimalist GNU for Windows"(Section \ref{sec:WinInstallMinGW}) and the C++ Compiler from Microsoft Visual Studio (Section \ref{sec:WinInstallVS}). 

\item Then, \textbf{process to Basic installation} of FA$\mu$ST tool by following instructions given in the appropriate section, depending to the kind of compiler (MinGW or Microsoft Visual) and the use (or not) of an IDE (Integrated Development Environment). 
\end{itemize}

Section \ref{sec:WinCustomInstall} describes the custom options available to build the FA$\mu$ST toolbox, to propose Custom - Advanced installation. For example, the optional configuration can be used to modify the install directory path, or to build in debug mode.  


\lstset{style=customBash}
\begin{lstlisting}
> This kind of script represents the text command you must 
> entered in an terminal command. For example : 
> mkdir BUILD 
\end{lstlisting}
\lstset{style=customBash}
\begin{lstlisting}
when character '>' is missing in the beginning of each line,
it signifies that the message is an return text in your
current terminal. 
\end{lstlisting}
\lstset{style=customMatlab}
\begin{lstlisting}
>> This kind of script represents the text command you must
>> entered in a Matlab Command Window. 
\end{lstlisting}
\lstset{style=customMatlab}
\begin{lstlisting}
when character ">>" is missing in the beginning of each line, 
it signifies that the message is a return text in your current 
Matlab Command Window. 
\end{lstlisting}


\section{Required components}\label{sec:WinRequired}
The installation of the FA$\mu$ST tool depends on other components to be installed in order to run properly. 

\begin{itemize}
\item \textbf{Install CMake}. The minimum version required is Cmake version 3.0.2. Download Binary distributions correspond to your environment from \url{https://cmake.org/download/}.
\item \textbf{Verify CMake install} : Open a terminal and type the following command:
\lstset{style=customBash}
\begin{lstlisting}
> where cmake
\end{lstlisting}
You should obtain the path of your \texttt{cmake.exe} binary file. If not, please add the directory of your \texttt{cmake.exe} file in your environment variable. (to add an environment variable, follow instructions given in \ref{sec:ANNEXEEnvironmentVariableWindows}). 

\item \textbf{Install Matlab} (minimal version required is Matlab 2014a) (\url{https://fr.mathworks.com/downloads/}). 

\item \textbf{Verify Matlab install} by checking the \texttt{matlab.exe} binary file in the default directory:  type in a terminal the following command : 
\lstset{style=customBash}
\begin{lstlisting}
> where matlab
\end{lstlisting}
You must obtain the path of your matlab binary file like: 
\begin{lstlisting}
C:\Program Files\MATLAB\<R2015b>\bin\matlab.exe
\end{lstlisting}
If not, please add the directory of your \texttt{matlab.exe} file in your environment variable. (to add an environment variable, follow instructions given in \ref{sec:ANNEXEEnvironmentVariableWindows}). \\

\item \textbf{Verify Matlab} has a MEX supported compiler by typing in a Matlab Command Window :
\lstset{style=customMatlab}
\begin{lstlisting}
>> mex -setup C++
\end{lstlisting}

You must obtain this kind of message :
\begin{lstlisting}
MEX configured to use <YOURCOMPILER>  for C++ language compilation.
\end{lstlisting}

But if you have an error message of the style :
\begin{lstlisting}[moredelim={**[is][\color{blue}]{@}{@}},moredelim={[is][\underbar]{_}{_}}]
No supported compiler or SDK was found. For options, visit 
@_http:www.matworks.com/support/compilers/R<20XXx>/<ARCH>.html_@
\end{lstlisting}
Visit the webpage specified in the error message,
in order to see the list of compiler supported by your Matlab version \texttt{R<20XXx>} on your plateform \texttt{<ARCH>}.

\item \textbf{Install 7-zip tool} from \url{http://www.7-zip.org/}. 

\item \textbf{Verify 7-zip install} by typing in a terminal the following command : 
\lstset{style=customBash}
\begin{lstlisting}
> where 7z
\end{lstlisting}
You must obtain the path of your \texttt{7z.exe} binary file: 
\begin{lstlisting}
C:\Program Files\7-Zip\7z.exe
\end{lstlisting}
If not, add \texttt{7z.exe} directory in your environment path. 
(to add an environment variable, follow instructions given in \ref{sec:ANNEXEEnvironmentVariableWindows}). 

\end{itemize}

\section{Download Faust Package}\label{sec:WinDownload}
When prerequisitie components listed in precedent section \ref{sec:WinRequired} are checked, you can get the FA$\mu$ST package.
\begin{itemize}
\item \textbf{Download} the FA$\mu$ST package on the website : \url{http://faust.gforge.inria.fr/}
\item \textbf{Unzip} the FA$\mu$ST package into your FA$\mu$ST directory. 
\lstset{style=customBash}
\begin{lstlisting}
> 7z x FAUST-Source-2.0.0.zip
\end{lstlisting}
\end{itemize}


\section{Install using GCC-MinGW compiler}\label{sec:WinInstallMinGW}

When prerequisite components listed in previous section \ref{sec:WinRequired} are checked, you must install \textbf{MinGW containing the C++ Compiler}. 
Then, the FA$\mu$ST installation can be done using Command prompt, following instructions given in Section \ref{sec:WinMinGWBasicInstall} or using CodeBlocks IDE, following instructions given in Section \ref{sec:WinCodeBlocksBasicInstall}. 


\subsection{Install GCC from MinGW}
\label{sec:WinInstallCompilerMinGW}

\begin{itemize}
\item \textbf{Download MinGW} in \url{https://sourceforge.net/projects/mingw/files/latest/download?source=files}
\item \textbf{Launch install file} and choose latest version of MinGW. Select the MinGW corresponding to gcc-g++ compiler. Then, select mingw32-make.exe in "All packages" menu. 
\item Add \texttt{gcc.exe} directory in your environment path. 
(to add an environment variable, follow instructions given in \ref{sec:ANNEXEEnvironmentVariableWindows}). 
\item \textbf{Verify MinGW install} by typing in a terminal the following command : 

\lstset{style=customBash}
\begin{lstlisting}
> where gcc
\end{lstlisting}
You must obtain the path of your \texttt{gcc.exe} binary file: 
\begin{lstlisting}
C:\mingw\mingw64\bin\gcc.exe
\end{lstlisting}

If not, add \texttt{gcc.exe} directory in your environment path. 
(to add an environment variable, follow instructions given in \ref{sec:ANNEXEEnvironmentVariableWindows}). 

\item \textbf{Verify make tool:} In a terminal command, type
\begin{lstlisting}
> where make
\end{lstlisting}
You must obtain the path of your \texttt{make.exe} binary file. 
If not, please check if \texttt{make.exe} file is present in MINGW install directory. If not, you can copy and rename \texttt{mingw32-make.exe} to \texttt{make.exe}.

\item From Matlab IDE, you must install MinGW version 4.9.2 using the \textbf{ADDON menu}. For more detail, please follow the instruction given in following link :  
\url{http://fr.mathworks.com/help/matlab/matlab_external/install-mingw-support-package.html}. For that, you must have an id session for Mathwork (easy to create). Current this latest step, an environment variable called MW\_MINGW64\_LOC is automatically generated. 
\end{itemize}


\subsection{Basic Build \& Install using the command prompt}
\label{sec:WinMinGWBasicInstall}

The basic install described in this section requires the GCC compiler from MinGW (see precedent Section \ref{sec:WinInstallCompilerMinGW}).
If you have administrator privilege, follow instructions in Section \ref{sec:WinMinGWadminBasicInstall}, otherwise follow instructions given in Section \ref{sec:WinMinGWNoAdminBasicInstall}.
\subsubsection{Install with administrator privilege}
\label{sec:WinMinGWadminBasicInstall}

\begin{itemize}
\item Open a command terminal
\item Set the current directory to your FA$\mu$ST directory (NOTE: don't use any special character in your FA$\mu$ST directory path, for example the character $\mu$)
\item Type the following commands : 

\lstset{style=customBash}
\begin{lstlisting}
> mkdir build
> cd build
> cmake -G "MinGW Makefiles" .. 
> make
\end{lstlisting}

\item Open a command terminal \textbf{with administrator privilege} : right click on command prompt icon and select run as administrator. 
\item Set the current directory to your FA$\mu$ST directory
\item Type the following commands : 
\begin{lstlisting}
> make install 
\end{lstlisting}
\end{itemize}

FA$\mu$ST Toolbox should be installed. Now, refer to Quick-Start Chapter \ref{sec:firstUse} to check the install and to try FA$\mu$ST toolbox.


\subsubsection{Install without administrator privilege}
\label{sec:WinMinGWNoAdminBasicInstall}

\begin{itemize}
\item Open a command terminal
\item Set the current directory to your FA$\mu$ST directory (NOTE: don't use any special character in your FA$\mu$ST directory path, for example the character $\mu$)
\item Type the following commands : 

\lstset{style=customBash}
\begin{lstlisting}
> mkdir build
> cd build
> cmake -G "MinGW Makefiles" .. 
	    -DCMAKE_INSTALL_PREFIX="<Your/Install/Dir>"
> make
> make install 
\end{lstlisting}

\end{itemize}
FA$\mu$ST Toolbox should be installed. Now, refer to Quick-Start Chapter \ref{sec:firstUse} to check the install and to try FA$\mu$ST toolbox.

\subsection{Basic build \& Install using Code::Blocks IDE}
\label{sec:WinCodeBlocksBasicInstall}

The basic install described in this section requires the GCC compiler from MinGW (see precedent Section \ref{sec:WinInstallCompilerMinGW}).
If you have administrator privilege, follow instructions in Section \ref{sec:WinMinGWCodeBlocksAdminBasicInstall}, otherwise follow instructions given in Section \ref{sec:WinMinGWCodeBlocksNoAdminBasicInstall}.

You must select the gcc compiler in the IDE code::Blocks in the \\ \texttt{Settings menu --> Compiler... --> Selected Compiler }  

Then, the FA$\mu$ST installation can be done. 

\subsubsection{Install with administrator privilege}
\label{sec:WinMinGWCodeBlocksAdminBasicInstall}
\begin{itemize}
\item Open a command terminal
\item Set the current directory to your FA$\mu$ST directory (NOTE: don't use any special character in your FA$\mu$ST directory path, for example the character $\mu$)
\item Type the following commands : 

\lstset{style=customBash}
\begin{lstlisting}
> mkdir build
> cd build
> cmake -G "CodeBlocks - MinGW Makefiles" .. 
\end{lstlisting}
\item Open the FA$\mu$ST project from the file \texttt{./build/FAUST.cbp} with Code::Blocks IDE.
\item In Code::Blocks IDE, select ALL BUILD, select RELEASE mode and generate the project.
\item Re-Open the FA$\mu$ST project from the file ./build/FAUST.cbp with Code::Blocks IDE \textbf{with administrator privilege}. For that, right-click on the Code::Blocks icon and select "run as administrator". 
\item In Code::Blocks IDE, select INSTALL target, select RELEASE mode and build.
\end{itemize}

FA$\mu$ST Toolbox should be installed. Now, refer to Quick-Start Chapter \ref{sec:firstUse} to check the install and to try FA$\mu$ST toolbox.

For more detail about cmake commands, refer to Section \ref{sec:ANNEXEInfoBuildInstall}.

\subsubsection{Install without administrator privilege}
\label{sec:WinMinGWCodeBlocksNoAdminBasicInstall}
\begin{itemize}
\item Open a command terminal
\item Set the current directory to your FA$\mu$ST directory (NOTE: don't use any special character in your FA$\mu$ST directory path, for example the character $\mu$)
\item Type the following commands : 

\lstset{style=customBash}
\begin{lstlisting}
> mkdir build
> cd build
> cmake -G "CodeBlocks - MinGW Makefiles" .. 
	    -DCMAKE_INSTALL_PREFIX="<Your/Install/Dir>"
\end{lstlisting}

\item Open the FA$\mu$ST project from the file \texttt{./build/FAUST.cbp} with Code::Blocks IDE.
\item In Code::Blocks IDE, select ALL target, select RELEASE mode and build the project.
\item In Code::Blocks IDE, select install target, select RELEASE mode and build the project.
\end{itemize}

FA$\mu$ST Toolbox should be installed. Now, refer to Quick-Start Chapter \ref{sec:firstUse} to check the install and to try FA$\mu$ST toolbox.

For more detail about cmake commands, refer to Section \ref{sec:ANNEXEInfoBuildInstall}.


% MICCORSOFT VISUAL STUDIO
\section{Install using Microsoft Visual compiler}\label{sec:WinInstallVS}

The installation on Windows Platform using Visual Studio IDE has been tested but there is compilation problems depending of the version of your Windows and your Visual Studio. \textbf{This install configuration is not guaranteed} but you can try and report any problem and/or suggestion on install-list of FA$\mu$ST project on \url{http://lists.gforge.inria.fr/pipermail/faust-install/}. 

When prerequisites listed in section \ref{sec:WinRequired} are checked, you must install \textbf{Microsoft Visual Studio} containing the C++ Compiler (Section \ref{sec:WinInstallCompilerVS}).
Then, the FA$\mu$ST installation can be done using Visual Studio IDE, following instructions given in Section \ref{sec:WinVisualStudioBasicInstall} or using Command prompt, following instructions given in Section \ref{sec:WinVisualStudioTerminalBasicInstall}. 


%If you are more familiar with the graphical user interface, prefer the \textbf{Microsoft Visual C++} compiler. Otherwise, if you are friendly with Unix tools and command line terminal, preferd \textbf{MinGW "Minimalist GNU for Windows"} C++ compiler. The version of the C++ compiler must be coherent with the version of your Matlab version. In this documentation, the version of our C++ compiler corresponds to Matlab 2014 and 2015. If you use an other version of Matlab, please refer to the Mathworks website \url{http://fr.mathworks.com/support/compilers/<R20XXa>}.

\subsection{Install Microsoft Visual compiler}\label{sec:WinInstallCompilerVS} 

\begin{itemize}
\item Download Microsoft Visual Studio Professional 2013 from \url{https://www.microsoft.com/en-US/download/details.aspx?id=44916}
\item Install Microsoft Visual Studio Professional 2013
% \item Download and install Microsoft .NET Framework 4
% \item Download and install Microsoft SDK 7.1
\end{itemize}
Then, the FA$\mu$ST installation can be done using Visual Studio IDE, following instructions given in Section \ref{sec:WinVisualStudioBasicInstall} or using Command prompt, following instructions given in Section \ref{sec:WinVisualStudioTerminalBasicInstall}. 

\subsection{Basic Build \& Install using Visual Studio IDE}\label{sec:WinVisualStudioBasicInstall}
The basic install described in this section requires Microsoft Visual Studio (see precedent Section \ref{sec:WinInstallCompilerVS}).
If you have administrator privilege, follow instructions in Section \ref{sec:AdminWinVisualStudioBasicInstall}, otherwise follow instructions given in Section \ref{sec:NoAdminWinVisualStudioBasicInstall}.

\subsubsection{Install with administrator privilege}
\label{sec:AdminWinVisualStudioBasicInstall}
\begin{itemize}
\item Open a command terminal
\item Set the current directory to your FA$\mu$ST directory (NOTE: don't use any special character in your FA$\mu$ST directory path, for example the character $\mu$)
\item Type the following commands :

\lstset{style=customBash} 
\begin{lstlisting}
> mkdir build
> cd build
> cmake .. 
\end{lstlisting}

\item Open the FA$\mu$ST project from the file \texttt{./build/FAUST.sln} with Visual Studio IDE.
\item In Visual Studio IDE, select ALL BUILD, select RELEASE mode, and generate the project three times.
\item Close Visual Studio IDE
\item Re-Open the FA$\mu$ST project from the file \texttt{./build/FAUST.sln} with Visual Studio IDE with \textbf{administrator privilege}.
\item In Visual Studio IDE, select INSTALL target, select RELEASE mode, and generate the project.
\end{itemize}

FA$\mu$ST Toolbox should be installed. Now, refer to Quick-Start Chapter \ref{sec:firstUse} to check the install and to try FA$\mu$ST toolbox.

For more detail about cmake commands, refer to Section \ref{sec:ANNEXEInfoBuildInstall}.

 
\subsubsection{Install without administrator privilege}
\label{sec:NoAdminWinVisualStudioBasicInstall}

\begin{itemize}

\item Open a command terminal
\item Set the current directory to your FA$\mu$ST directory (NOTE: don't use any special character in your FA$\mu$ST directory path, for example the character $\mu$)
\item Type the following commands :

\lstset{style=customBash} 
\begin{lstlisting}
> mkdir build
> cd build
> cmake .. -DCMAKE_INSTALL_PREFIX="<Your/Install/Dir>"
\end{lstlisting}

\item Open the FA$\mu$ST project from the file \texttt{./build/FAUST.sln} with Visual Studio IDE.
\item In Visual Studio IDE, select ALL target, select RELEASE mode and generate the project three times.
\item In Visual Studio IDE, select install target, select RELEASE mode, and generate the project.
\end{itemize}

FA$\mu$ST Toolbox should be installed. Now, refer to Quick-Start Chapter \ref{sec:firstUse} to check the install and to try FA$\mu$ST toolbox.

For more detail about cmake commands, refer to Section \ref{sec:ANNEXEInfoBuildInstall}.

\subsection{Basic Build \& Install using terminal}\label{sec:WinVisualStudioTerminalBasicInstall}
The basic install described in this section requires the Microsoft Visual Studio compiler (see precedent Section \ref{sec:WinInstallCompilerVS}).
If you have administrator privilege, follow instructions in Section \ref{sec:AdminWinVisualStudioTerminalBasicInstall}, otherwise follow instructions given in Section \ref{sec:NoAdminWinVisualStudioTerminalBasicInstall}.

\subsubsection{Install with administrator privilege}
\label{sec:AdminWinVisualStudioTerminalBasicInstall}
In the case of \textbf{Microsoft Visual Studio 2013 compiler using the command terminal} :
\begin{itemize}
\item Open a command terminal
\item Set the current directory to your FA$\mu$ST directory (NOTE: don't use any special character in your FA$\mu$ST directory path, for example the character $\mu$)
\item Type the following commands : 

\lstset{style=customBash}
\begin{lstlisting}
> mkdir build
> cd build
> cmake ..
> cmake --build . --config "Release"
> cmake --build . --config "Release"
\end{lstlisting}

\item Re-Open a command terminal \textbf{with administrator privilege} 
\lstset{style=customBash}
\begin{lstlisting}
> cmake --build . --config "Release" --target "install"
\end{lstlisting}
\end{itemize}

FA$\mu$ST Toolbox should be installed. Now, refer to Quick-Start Chapter \ref{sec:firstUse} to check the install and to try FA$\mu$ST toolbox.

For more detail about cmake commands, refer to Section \ref{sec:ANNEXEInfoBuildInstall}.

\subsubsection{Install without administrator privilege}
\label{sec:NoAdminWinVisualStudioTerminalBasicInstall}

In the case of \textbf{Microsoft Visual Studio 2013 compiler using the command terminal} :
\begin{itemize}
\item Open a command terminal
\item Set the current directory to your FA$\mu$ST directory (NOTE: don't use any special character in your FA$\mu$ST directory path, for example the character $\mu$)
\item Type the following commands : 
\end{itemize}
\lstset{style=customBash}
\begin{lstlisting}
> mkdir build
> cd build
> cmake .. -DCMAKE_INSTALL_PREFIX="<Your/Install/Dir>" 
> cmake --build . --config "Release"
> cmake --build . --config "Release"
> cmake --build . --config "Release" --target "install"
\end{lstlisting}


FA$\mu$ST Toolbox should be installed. Now, refer to Quick-Start Chapter \ref{sec:firstUse} to check the install and to try FA$\mu$ST toolbox.

For more detail about cmake commands, refer to Section \ref{sec:ANNEXEInfoBuildInstall}.



\section{Custom - Advanced Installation}\label{sec:WinCustomInstall}

The project FA$\mu$ST can be configured with optional parameters, for example if you want to install FA$\mu$ST library in a different folder or to enable the parallel computing using multithread capacities provided by the OS. This build system can be parametrized using the Cmake Graphical User Interface, or the Cmake command line tools. 
The Cmake Graphical User Interface allows you to select option input. 

\begin{enumerate}
\item Open application \texttt{cmake-GUI.exe} from the program menu or from your cmake install binaries directory  to launch the CMake configuration application:

\begin{figure}[H] %%[!htbp]
\centering
\includegraphics[scale=0.5]{images/cmakeGUI-1-eps-converted-to.pdf}
\caption{cmake GUI}
\label{fig:cmakeGUI-1}
\end{figure}

\item Set the "Where is the source code:" text box with the path of the directory where the source files are located (F:/WORK/FAUST/faust) and the "Where to build the binaries:" with the path of the directory where you want to build the library and executable files (F:/WORK/FAUST/faust/build). (see fig.  \ref{fig:cmakeGUI-1}).

When clicking for the first time on the [Configure] button, CMake will ask for the build tool you want to use. The build system type depends on the builder you want to use, in our case this is the Visual Studio X (X depending the version of Visual installed on the computer) chain tools. (see fig. \ref{fig:cmakeGUI-2}).


\begin{figure}[H]
\centering
\includegraphics[scale=0.5]{images/cmakeGUI-2-eps-converted-to.pdf}
\caption{cmake GUI}
\label{fig:cmakeGUI-2}
\end{figure}

\item When pressing again the [Configure] button to configure the build system, CMake performs a list of tests to determine the system configuration and manage the build system. If the configuration is correct then no pop-up will appears during the tests and CMake finally shows the various options of the build underlaid in grey. In case of a configuration issue, a pop up window warns you about this issue indicating which test has failed, in this case the build option in the CMake application software will be underlaid in red. We will discuss in Section \ref{sec:WinCustomInstall} what to do in such a case, but let us for the moment assume that everything ran smoothly.
(see \ref{fig:cmakeGUI-4}).

\begin{figure}[H] %%[!htbp]
\centering
\includegraphics[scale=0.5]{images/cmakeGUI-4-eps-converted-to.pdf}
\caption{cmake GUI}
\label{fig:cmakeGUI-4}
\end{figure}

\end{enumerate}

Here is the list of available options: 
\begin{itemize}
\item CMAKE\_INSTALL\_PREFIX : Install directory
\item BUILD\_TESTING : Enable the ctest option (default value is ON)
\item BUILD\_DOCUMENTATION : Generating the doxygen documentation (default value is OFF)  
\item BUILD\_MULTITHREAD : Enable multithread with OpenMP Multithreading (default value is OFF)
\item BUILD\_VERBOSE : Enable verbose option when compile (-v) (default value is OFF)
\item BUILD\_DEBUG : Enable FA$\mu$ST Debug mode (default value is OFF )
\item BUILD\_USE\_GPU : Using both CPU and GPU process ( default value is OFF) (refer to Annex \ref{sec:OptionalGPU} for installation and more detail)
\item BUILD\_MATLAB\_MEX\_FILES : Enable building Matlab MEX files (default value is ON)
\item BUILD\_OPENBLAS : Using openBLAS for matrix and vector computations (default value is OFF )
%\item BUILD\_READ\_XML\_FILE : Using xml2 library to read xml files (default value is OFF)
%\item BUILD\_READ\_MAT\_FILE : Using matio library to read mat files (default value is OFF)

\end{itemize}




%%%%AUTRESS


