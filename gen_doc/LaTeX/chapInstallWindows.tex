\chapter{Installation on Windows platform}\label{sec:WinInstall}


\paragraph{}The FA$\mu$ST project is based on \textbf{C++ library} available for both UNIX and Windows environments. The proposed toolbox provides a Matlab wrapper. \textbf{CMake} has been chosen to build the project FA$\mu$ST because it is an open-source, cross-platform family of tools designed to build, test and package software. This chapter presents the steps to install the FA$\mu$ST tools on Windows platform.

\begin{itemize}
\item Firstly, please ensure that the \textbf{prerequisites components} listed in Section \ref{sec:WinRequired} are installed. 
\item Secondly, choose and install your preferred compiler between GCC from MinGW (Section \ref{sec:WinInstallCompilerMinGW}) or compiler from microsoft visual studio (Section \ref{sec:WinInstallCompilerVS}).
\item Then, process to Basic installation using \textbf{the command terminal} (refer to Section \ref{sec:WinBasicInstall}) or Basic installation using \textbf{the IDE "Visual Studio"} (refer to Section \ref{sec:WinVisualStudioBasicInstall}).
\end{itemize}


%Then refer to the appropriate section : If you are more familiar with the graphical user interface, prefer the basic installation using \textbf{"the IDE Visual Studio"}. Otherwise, if you are friendly with Unix tools and command line terminal, prefer the basic installation using \textbf{the command terminal}: 


Section \ref{sec:WinCustomInstall} presents the configure options available to build the FA$\mu$ST toolbox are described to propose an Custom - Advanced installation. For example, the optional configuration can be used to modify the install directory path, or to build in debug mode.  

%\textbf{Microsoft visual C++} compiler using the IDE 
%\textbf{MinGW "Minimalist GNU for Windows"} C++ compiler using


\section{Required components}\label{sec:WinRequired}
The installation of the FA$\mu$ST tool depends on other components to be installed in order to run properly. 

\begin{itemize}
\item \textbf{Install CMake}: Download Binary distributions correspond to your environment from \url{https://cmake.org/download/}.
\item \textbf{Verify CMake install} : Open a terminal and type the following command:
\begin{lstlisting}
where cmake
\end{lstlisting}
You should obtain the path of your \texttt{cmake.exe} binary file. If not, verify that the CMAKE install directory is without any space character. Otherwise, please add the directory of your \texttt{cmake.exe} file in your environment variable. (to add an environment variable, follow instructions given in \ref{sec:ANNEXEEnvironmentVariableWindows}). 

\item \textbf{Install Matlab} (\url{https://fr.mathworks.com/downloads/}). You must install Matlab in the default directory like:
\begin{lstlisting} 
C:\Program Files\MATLAB
C:\Program Files (x86)\MATLAB
\end{lstlisting}
or install Matlab in a directory without any space character.

\item \textbf{Verify Matlab install} by checking the \texttt{matlab.exe} binary file in the default directory, for example in  
\begin{lstlisting} 
C:\Program Files\MATLAB\<R2015b>\bin\matlab.exe
\end{lstlisting}
In the case of an installation in a directory without any space character, type in a terminal the following command : 
\begin{lstlisting}
where matlab
\end{lstlisting}
You must obtain the path of your matlab binary file like: 
\begin{lstlisting}
C:\ProgramFiles\MATLAB\<R2015b>\bin\matlab.exe
\end{lstlisting}
If not, please add the directory of your \texttt{matlab.exe} file in your environment variable. (to add an environment variable, follow instructions given in \ref{sec:ANNEXEEnvironmentVariableWindows}). 


\item \textbf{Install 7-zip tool} from \url{http://www.7-zip.org/} in a install directory without any space character. 
\item \textbf{Verify 7-zip install} by typing in a terminal the following command : 
\begin{lstlisting}
where 7z
\end{lstlisting}
You must obtain the path of your \texttt{7z.exe} binary file: 
\begin{lstlisting}
C:\program\7-Zip\7z.exe
\end{lstlisting}
If not, verify that the 7-Zip install directory is without any space character. Otherwise, add \texttt{7z.exe} directory in your environment path. 
(to add an environment variable, follow instructions given in \ref{sec:ANNEXEEnvironmentVariableWindows}). 

\end{itemize}


\subsection{Compiler Install}\label{sec:WinInstallCompiler}
\paragraph{}Both \textbf{Microsoft visual C++} from visual studio and \textbf{GCC} from MinGW "Minimalist GNU for Windows" compilers have been tested. 

%If you are more familiar with the graphical user interface, prefer the \textbf{Microsoft visual C++} compiler. Otherwise, if you are friendly with Unix tools and command line terminal, preferd \textbf{MinGW "Minimalist GNU for Windows"} C++ compiler. The version of the C++ compiler must be coherent with the version of your Matlab version. In this documentation, the version of our C++ compiler corresponds to Matlab 2014 and 2015. If you use an other version of Matlab, please refer to the Mathworks website \url{http://fr.mathworks.com/support/compilers/<R20XXa>}.

\subsubsection{Microsoft visual C++}\label{sec:WinInstallCompilerVS} \begin{itemize}
\item Download Microsoft Visual Studio Professional 2013 from \url{https://www.microsoft.com/en-US/download/details.aspx?id=44916}
\item Install Microsoft Visual Studio Professional 2013
% \item Download and install Microsoft .NET Framework 4
% \item Download and install Microsoft SDK 7.1
\end{itemize}
Then, the FA$\mu$ST installation can be done with Visual Studio IDE, following instructions given in Section \ref{sec:WinVisualStudioBasicInstall}. 


\subsubsection{GCC from MinGW}
\label{sec:WinInstallCompilerMinGW}

\paragraph{}When prerequisites listed in previous section \ref{sec:WinRequired} are checked, you must install \textbf{MinGW containing the C++ Compiler}

\begin{itemize}
\item \textbf{Download MinGW} in \url{https://sourceforge.net/projects/mingw/files/latest/download?source=files}
\item \textbf{Launch install file} and choose MinGW version 4.9.2 for mex tool compatibility. The install directory must be \textbf{without any space character}.  
\item Add \texttt{gcc.exe} directory in your environment path. 
(to add an environment variable, follow instructions given in \ref{sec:ANNEXEEnvironmentVariableWindows}). 
\item \textbf{Verify MinGW install} by typing in a terminal the following command : 
\begin{lstlisting}
where gcc
\end{lstlisting}
You must obtain the path of your \texttt{gcc.exe} binary file: 
\begin{lstlisting}
C:\mingw-w64\mingw64\bin\gcc.exe
\end{lstlisting}
If not, add \texttt{gcc.exe} directory in your environment path. 
(to add an environment variable, follow instructions given in \ref{sec:ANNEXEEnvironmentVariableWindows}). 

\item \textbf{Verify make tool:} In a terminal command, type
\begin{lstlisting}
where make
\end{lstlisting}
You must obtain the path of your \texttt{make.exe} binary file. 
If not, verify that the MINGW install directory is without any space character. Otherwise, please check if \texttt{make.exe} file is present in MINGW install directory. If not, you can copy and rename \texttt{mingw32-make.exe} to \texttt{make.exe}.

\item From Matlab IDE, you must install MinGW version 4.9.2 using the \textbf{ADDON menu}. For more detail, please follow the instruction given in following link :  
\url{http://fr.mathworks.com/help/matlab/matlab_external/install-mingw-support-package.html}. For that, you must have a id session for Mathwork (easy to create). Current this latest step, an environment variable called MW\_MINGW64\_LOC is automatically generated. 
\end{itemize}
Then, the FA$\mu$ST installation can be done with :
\begin{itemize}
\item Command prompt, following instructions given in Section \ref{sec:WinMinGWBasicInstall} 
\item CodeBlocks IDE, following instructions given in Section \ref{sec:WinCodeBlocksBasicInstall} 
\end{itemize}



\section{Basic Build \& Install using the command prompt}
\label{sec:WinMinGWBasicInstall}
This section concerns the GCC compiler from MinGW install.
\subsection{Install with administrator privilege}
\label{sec:WinMinGWadminBasicInstall}
\begin{itemize}
\item \textbf{download the FA$\mu$ST package} on the website  \url{http://faust.gforge.inria.fr/}. 
\item \textbf{Unzip} the FA$\mu$ST package into your FA$\mu$ST directory. 

\item Open a command terminal
\item Set the current directory to your FA$\mu$ST directory (NOTE: don't use any special character in your FAUST directory path, for example the character $\mu$)
\item Type the following commands : 
\begin{lstlisting}
mkdir build
cd build
cmake -G "MinGW Makefiles" .. 
make
\end{lstlisting}

\item Open a command terminal \textbf{with administrator privilege} : right click on command prompt icon and select run as administrator. 
\item Set the current directory to your FA$\mu$ST directory
\item Type the following commands : 
\begin{lstlisting}
make install 
\end{lstlisting}
\end{itemize}

\subsection{Install without administrator privilege}
\label{sec:WinMinGWNoAdminBasicInstall}

\begin{itemize}
\item \textbf{download the FA$\mu$ST package} on the website  \url{http://faust.gforge.inria.fr/}. 
\item \textbf{Unzip} the FA$\mu$ST package into your FA$\mu$ST directory. 

\item Open a command terminal
\item Set the current directory to your FA$\mu$ST directory (NOTE: don't use any special character in your FAUST directory path, for example the character $\mu$)
\item Type the following commands : 
\begin{lstlisting}
mkdir build
cd build
cmake -G "MinGW Makefiles" .. 
	  - DCMAKE\_INSTALL\_PREFIX="<Your/Install/Dir>"
make
make install 
\end{lstlisting}
\end{itemize}


\section{Basic build \& Install using Code::Blocks IDE}
\label{sec:WinCodeBlocksBasicInstall}
This section concerns the GCC compiler from MinGW install.
\paragraph{}When prerequisites listed in previous section \ref{sec:WinRequired} are checked, you must install \textbf{MinGW containing the C++ Compiler}. For MinGW install, please refer to precedent Section \ref{sec:WinBasicInstall}). 
You must select the gcc compiler in the IDE codeBlocks in the \\ \texttt{Settings menu --> Compiler... --> Selected Compiler }  

Then, the FA$\mu$ST installation can be done. 
\begin{itemize}
\item \textbf{download the FA$\mu$ST package} on the website  \url{http://faust.gforge.inria.fr/}. 
\item \textbf{Unzip} the FA$\mu$ST package into your FA$\mu$ST directory. 

\item Open a command terminal
\item Set the current directory to your FA$\mu$ST directory (NOTE: don't use any special character in your FAUST directory path, for example the character $\mu$)
\item Type the following commands : 
\begin{lstlisting}
mkdir build
cd build
cmake -G "CodeBlocks - MinGW Makefiles" .. 
\end{lstlisting}
\item Open the FA$\mu$ST project from the file \texttt{./build/FAUST.cbp} with Code::Blocks IDE.
\item In Code::Blocks IDE, select ALL target and build the project.
\item In Code::Blocks IDE, select install target and build the project.
For more detail about cmake commands, refer to Section \ref{sec:ANNEXEInfoBuildInstall}.
\end{itemize}


\section{Basic Build \& Install Using Visual Studio IDE}\label{sec:WinVisualStudioBasicInstall}
This section concerns the use of Microsoft Visual Studio compiler.

\paragraph{}In the case of \textbf{Microsoft Visual Studio 2013 compiler using the Graphical Users Interfaces}:

\begin{enumerate}
\item Once the build system configured then generated, you have to actually build FAUST, using Visual Studio.
\item Open file "faust.sln" with visual studio 
\item Click right on Target ALL\_BUILD and select generated 
\item Click right on Target INSTALL and select generated 
\item Click right on Target CTEST and select generated 
\end{enumerate}


\bigbreak
\paragraph{NOTE:}In the case of \textbf{Microsoft Visual Studio 2013 compiler using the command terminal} :

\begin{itemize}
\item Open a command terminal
\item Set the current directory to your FA$\mu$ST directory (NOTE: don't use any special character in your FAUST directory path, for example the character $\mu$)
\item Type the following commands : 

\begin{lstlisting}
mkdir build
cd build
cmake .. 
cmake --build . --config "Release" --target "install"
\end{lstlisting}

\end{itemize}


\section{Custom - Advanced Installation}\label{sec:WinCustomInstall}
\begin{enumerate}
\item Open application \texttt{cmake-GUI.exe} from the program menu or from your cmake install binaries directory  to launch the CMake configuration application:

\begin{figure}[!h] %%[!htbp]
\centering
\includegraphics[scale=0.5]{images/cmakeGUI-1-eps-converted-to.pdf}
\caption{cmake GUI}
\label{fig:cmakeGUI-1}
\end{figure}

\item Set the "Where is the source code:" text box with the path of the directory where the source files are located (F:/WORK/FAUST/faust) and the "Where to build the binaries:" with the path of the directory where you want to build the library and executable files (F:/WORK/FAUST/faust/build). (see fig.  \ref{fig:cmakeGUI-1}).

When clicking for the first time on the [Configure] button, CMake will ask for the build tool you want to use. The build system type depends on the builder you want to use, in our case this is the Visual Studio X (X depending the version of Visual installed on the computer) chain tools. (see fig. \ref{fig:cmakeGUI-2}).


\begin{figure}[!h]
\centering
\includegraphics[scale=0.5]{images/cmakeGUI-2-eps-converted-to.pdf}
\caption{cmake GUI}
\label{fig:cmakeGUI-2}
\end{figure}

\item When pressing again the [Configure] button to configure the build system, CMake performs a list of tests to determine the system configuration and manage the build system. If the configuration is correct then no pop-up will appears during the tests and CMake finally shows the various options of the build underlaid in grey. In case of a configuration issue, a pop up window warns you about this issue indicating which test has failed, in this case the build option in the CMake application software will be underlaid in red. We will discuss in Section \ref{sec:WinCustomInstall} what to do in such a case, but let us for the moment assume that everything ran smoothly.
(see \ref{fig:cmakeGUI-4}).

\begin{figure}[!h] %%[!htbp]
\centering
\includegraphics[scale=0.5]{images/cmakeGUI-4-eps-converted-to.pdf}
\caption{cmake GUI}
\label{fig:cmakeGUI-4}
\end{figure}

\end{enumerate}

%%%%AUTRESS


